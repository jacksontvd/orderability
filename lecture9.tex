\chapter{Orderings of the braid group}
\marginnote{Lecture 9 (Hannah Turner); February 18, 2020}

We will follow \cite{braids}.

Let $z_1 , \ldots , z_n\in \DD^2$. A \emph{braid on $n$ strands} is a subset $\b \sub
\DD^2 \times I$ such that $\b$ is a union of smoothly embedded intervals (called
\emph{strands}) in $\DD^2 \times I$ such that
\begin{enumerate}
\item $\b \cap \left(D^2 \times \left\{1\right\}\right) = \left\{\left(z_1 , 1\right) ,
\ldots , \left(z_n , 1\right)\right\}$,
\item $\b\cap \left(D^2 \times \left\{0\right\}\right) = \left\{\left(z_1,0\right) , \ldots
, \left(z_n , 0\right)\right\}$,
\item $\b \trans \left(\DD^2 \times \left\{t\right\}\right)$ in $n$ points.
\end{enumerate}

We should think of braids as these strands weaving around one another
as in \cref{fig:braid}.

\begin{figure}
\begin{overlay}
\pict{braid.pdf}{0.5}
\end{overlay}
\caption{A braid on $3$ strands.}
\label{fig:braid}
\end{figure}

We say two braids are equivalent if there is a deformation from one to the other through
braids. There is an operation on braids called \emph{stacking}. This takes two braids and
stacks them to make a new braid.

\begin{thm}[Artin]
The set of $n$-strand braids form a group $B_n$ with group operation given by stacking. 
In particular, it has the following presentation:
\begin{equation}
B_n = \lr{\sigma_1 , \ldots , \sigma_{n-1} \st 
\begin{matrix}
\abs{i-j} > 1 \implies \sigma_i \sigma_j = \sigma_j \sigma_i \ ,\\
\sigma_i \sigma_{i+1} \sigma_i = \sigma_{i+1}\sigma_i \sigma_{i+1}
\end{matrix}
}\ .
\end{equation}
\end{thm}

Geometrically, the generators $\sigma_i$ correspond to braids as in 
\cref{fig:generators}.
Now a braid $\b$ is an equivalence class of words in the $\sigma_i$.

\begin{figure}
\begin{overlay}
\pict{generators.pdf}{0.1}
\toptext{$i$}{-0.8,2}
\toptext{$i+1$}{1,2}
\toptext{$\ldots$}{-2,0}
\toptext{$\ldots$}{2,0}
\end{overlay}
\caption{The generator $\sigma_i$ of $B_n$.}
\label{fig:generators}
\end{figure}


There is a map $B_n \to \MCG\left(D_n\right)$ from the braid group to the mapping class
group of $D_n$, i.e. the group of orientation preserving homeomorphisms of $\DD^2$ with
$n$ punctures such that the punctures are fixed setwise, and $\p \DD^2$ is fixed
pointwise.
The map sends the generators 
\begin{equation*}
\sigma_i \mapsto h_{\sigma_i}: D_n\acted
\end{equation*}
to half-Dehn twists about the straight arc connecting $z_i$ and $z_{i+1}$.
See \cref{fig:dehn}.

\begin{figure}
\begin{overlay}
\pict{dehn.pdf}{0.8}
\toptext{$\lto{h_{\sigma_i}}$}{0,0}
\toptext{$z_i$}{-4,0.4}
\toptext{$z_{i+1}$}{-2.5,0.4}
\toptext{$z_{i+1}$}{3,-0.38}
\toptext{$z_{i}$}{3.8,0.35}
\end{overlay}
\caption{The half-Dehn twist about the straight arc connecting $z_i$ and $z_{i+1}$.}
\label{fig:dehn}
\end{figure}

\begin{clm}
This map is an isomorphism.
\end{clm}

\section{Dehornoy's ordering}

\begin{defn}
A braid word $w$ is said to be $\sigma$-positive (resp. $\sigma$-negative) if,
among the letters $\sigma^{\pm 1}_i$ that occur in $w$, the one with lowest index occurs
with only positive (resp. negative) exponent, i.e. $\sigma_i$ occurs but not $\sigma_i^{-1}$.
In this case we say
$w$ is $\sigma_i$ positive.
\end{defn}

\begin{rmk}
Usually we don't care for which $i$ the word is $\sigma_i$ positive. In this scenario we
just say $\om$ is $\sigma$-positive.
\end{rmk}

\begin{exm}
$\sigma_1 \sigma_2$ and $\sigma_1 \sigma_2^{-1}$ are both $\sigma_1$ positive.
$\sigma_1^{-1}\sigma_2$ is $\sigma_1$-negative.
\end{exm}

\begin{wrn}
Some braids are neither, e.g. $\sigma_2^{-1} \sigma_3 \sigma_2$.
\end{wrn}

\begin{defn}
We say $1<_{Deh} \b$ if $\b$ is $\sigma$-positive.
\end{defn}

Note $\b_1 <_{Deh} \b_2$ iff $1 <_{Deh} \b_1 \b_2$.

\begin{thm}[Dehornoy]
The above definition for
$<_{Deh}$ defines an LO on $B_n$.
\end{thm}

\begin{proof}[Proof idea]
We use the following properties to prove the theorem.
\begin{itemize}
\item Property A (Acyclicity): a $\sigma$-positive word is always nontrivial.
\item Property C (Comparison): Every nontrivial braid of $B_n$ admits an $n$-strand
representative word that is $\sigma$-positive or $\sigma$-negative.
\end{itemize}

Write $P_n$ for the positive braids on $n$-strands. We will show that $P_n$ is a positive
cone. 

\begin{enumerate}
\item $P_n$ is closed: let $\b_1 , \b_2 \in B_n$. If $\b_1$ is $\sigma_i$-positive, $\b_2$
is $\sigma_j$ positive for $i \leq j$. Then $\b_1 \b_2$ is $\sigma_i$ positive. For example:
\begin{align}
\b_1 &= \sigma_1\sigma_2 \sigma_3 \sigma_2^{-1} \\
\b_2 &= \sigma_2 \sigma_3 \sigma_2 \sigma_3^{-1} \\
\b_1 \b_2 &= \sigma_1\sigma_2 \sigma_3\sigma_3 \sigma_2\sigma_3^{-1}
\ .
\end{align}

\item $B_n\minus \left\{1\right\} = P_n \un P_n^{-1}$: property A implies
$1\not\in P_n$ and then property C implies this.

\item Disjoint union: Suppose $\b \in P_n \cap P_n^{-1}$. Then $\b^{-1}\in P_n$, so $\b
\b^{-1} = 1\in P_n$ which is a contradiction.
\end{enumerate}
\end{proof}

\begin{prop}
$B_n$ for $n\geq 3$ is not BO.
\end{prop}

\begin{proof}
Define 
\begin{equation*}
\Del_n = \left(\sigma_1 \ldots \sigma_{n-1}\right) \left(\sigma_1 \ldots
\sigma_{n-2}\right) \ldots \left(\sigma_1 \sigma_2\right)\sigma_1
\ .
\end{equation*}
For example, see 
\cref{fig:del} for $\Del_4$.
\begin{figure}
\begin{overlay}
\pict{del.pdf}{0.3}
\end{overlay}
\caption{The braid $\Del_4$.}
\label{fig:del}
\end{figure}
\begin{clm}
$\Del_n \sigma_i = \sigma_{n-i} \Del_n$.
\end{clm}
Now suppose $\cless$ is a BO on $B_n$. WLOG $\sigma_1 \cless \sigma_{n-1}$ implies
\begin{equation*}
\ubr{\Del_n \sigma_1 \Del_n^{-1}}{\sigma_{n-1}} \cless \ubr{\Del_n \sigma_{n-1}
\Del_n^{-1}}{\sigma_1}
\end{equation*}
so $\sigma_{n-1} \cless \sigma_1$, so
\begin{equation*}
\Del_n \sigma_i \Del_n^{-1} = \sigma_{n-i} \Del_n \Del_n^{-1} = \sigma_{n-i}
\end{equation*}
which is a contradiction.
\end{proof}

\begin{rmk}
\begin{enumerate}
\item For each $n$, two elements of
$\left(B_n , <_{Deh}\right)$ can be compared in polynomial
time (in the length of words). 
\item This ordering has applications to knot theory. If $\b \in B_n$ and $\b <
\Del_n^{-6}$ or $\b > \Del_n^g$, then its closure $\hat \b$ is prime.
\end{enumerate}
\end{rmk}

\begin{defn}
\begin{enumerate}
\item An LO group $\left(G , <\right)$ is \emph{Conradian} if for all $g,h > 1$, there is some
$p\in \ZZ^+$ with $h < gh^p$.
\item $\left(G , <\right)$ is \emph{Archimedean} if for all $g,h > 1$, there is $p\in
\ZZ^+$ with $g < h^p$.
\end{enumerate}
\end{defn}

\begin{prop}
$\left(B_n , <_{Deh}\right)$ is not Conradian nor Archimedean.
\end{prop}

% This is disturbing since it means there are $\al,\b>1$ such that
% for all $p$, $\b > \al \b^p$.

\section{Nielsen-Thurston orderings on \texorpdfstring{$B_n$}{the braid group}}

\begin{defn}
Suppose $G\acts \RR$ by orientation preserving homeomorphisms and there is $x\in \RR$ with
$\Stab_G\left(x\right) = \left\{1\right\}$. Then $\left(G , <_x\right)$ is defined by
declaring $g<_x g'$ iff $g\left(x\right) <_\RR g'\left(x\right)$.
\end{defn}

\begin{rmk}
\begin{enumerate}
\item This is an LO since $G < \homeo^+\left(\RR\right)$. 
\item Using $y\in \RR$, $y\neq x$ could give a different ordering.
\end{enumerate}
\end{rmk}

The goal is to get an action $B_n\acts \RR$.

We can give $D_n$
a hyperbolic metric. $\ttilde{D_n}$ is a subset of $\HH^2$. Now
compactify $\HH^2$ by adding $S^1_\infty$. Compactify $\ttilde {D_n}$ by adding in limit
points of lifts of $\p D_n$. This is a closed disk $\ttilde D_n$.
$\p \ttilde{D_n}$ has two types of points:
\begin{enumerate}
\item limit points, and 
\item arcs which cover $\p D_n$.
\end{enumerate}

Now pick a basepoint $\star$. For each $b\in B_n$, take $\b \mapsto h_\b: D_n\acted$. Note that
$h_\b$ has many lifts in $\ttilde{D_n}$. Pick one $\ttilde h_b$ that fixes the basepoint.
Now since $\p \ttilde{D_n} \minus \left\{\star\right\}\simeqq \RR$, we can restrict
$\tilde h_\b$ to $\p\ttilde{D_n} \minus \left\{\star\right\}$ to get an action on $\RR$.
Then it turns out this is all well-defined.

\begin{defn}
An LO $<$ on $B_n$ is of \emph{Nielsen-Thurston type} if there is some $x\in \RR$ such
that for all $\b$, $\b'\in B_n$ $\b < \b'$ iff $\b\left(x\right) <_\RR \b'\left(x\right)$.
\end{defn}

\begin{fact}
\begin{enumerate}
\item Some choices $x\in \RR$ have non-trivial stabilizer. These cannot give an ordering. 
\item Some choices $x\neq y\in \RR$ give the same ordering.
\item Uncountably many of them are distinct.
\end{enumerate}
\end{fact}

\section{Isolated orderings}

Recall LO's on $G$ correspond to positive cones. 

\begin{defn}
An ordering $<$ in $\LO\left(G\right)$ is \emph{finitely determined} if there is a finite
subset $S = \left\{g_1 , \ldots , g_k\right\}\sub G$ such that $<$ is the unique LO on $G$
such that $S$ is positive.
\end{defn}

\begin{exm}
\begin{enumerate}
\item $\left(\ZZ , <\right)$ is determined by choosing $\left\{1\right\}\sub P$.
\item If $P\sub G$ is finitely generated as a semi-group then the order $<$ determined by 
$P$ is finitely determined.
\item $K = \lr{a,b \st aba^{-1} = b^{-1}}$ is determined by $\left\{a,b\right\}$.
\end{enumerate}
\end{exm}

\begin{prop}
A points in $\LO\left(G\right)$ is isolated iff $<$ is finitely determined.
\end{prop}

\begin{proof}
$\left(\converse\right)$: Suppose that $< \in \LO\left(G\right)$ is finitely determined by
$f_1 , \ldots ,f_m$.
Recall $\LO\left(G\right)\sub \left\{0,1\right\}^G$. 
A basis for the topology is given by sets of the form:
\begin{equation}
B = \left\{ \left(\ubr{g_1 , \ldots , g_k}{\text{yes}} , \ubr{h_1 , \ldots ,
h_l}{\text{no}} , \ubr{\ldots}{\text{whatever}}\right)\right\}
\cap \LO\left(G\right) \ .
\label{eqn:B}
\end{equation}
Now we can impose that
\begin{enumerate}
\item The set of $g\in G$ which we say ``yes'' to is closed,
\item never say ``yes'' to both $g$ and $g^{-1}$
\item never say ``no'' to $g$ and $g^{-1}$.
\end{enumerate}
Then for
\begin{equation}
U = \left\{\left(f_1 , \ldots , f_m , f_1^{-1} , \ldots , f_m , \ldots\right)\right\}
\end{equation}
there is no other order inside $U$, so $<$ is isolated.

$\left(\implies\right)$: Assume $< \in \LO\left(G\right)$ is isolated. There is an open
set $U$ such that $<$ is the only element of $\LO\left(G\right)$. Write $<\in B\sub U$
where $B$ is of the form \eqref{eqn:B}. Then 
\begin{equation}
P\sups \left\{g_1 , \ldots , g_k , h_1^{-1} , \ldots , h_l^{-1}\right\}
\end{equation}
so $<$ is finitely determined.
\end{proof}

\begin{defn}[\cite{dubrov}]
Let $P_{DD}$ be the set of $\b\in B_3$ such that $\b$ is $\sigma_1$-positive or
$\sigma_2$-negative.
\end{defn}

\begin{thm}
$P_{DD}$ is a positive cone, and is generated as a semigroup by $\sigma_1 \sigma_2$ and
$\sigma_2^{-1}$.
\end{thm}

\begin{proof}
We will assume that a $\sigma_i$-positive word is never trivial.
We will also assume that either $\b$ is $\sigma_1$-positive or $\sigma_1$-negative or
$\sigma_1$-free.
Note that this implies $\sigma_1$-free braids are always $\sigma_2$-positive or
$\sigma_2$-negative.

Now we show $P_{DD}$ is a positive cone.
Write $Q = \lr{\sigma_1 \sigma_2 , \sigma_2^{-1}}$. This is a semigroup. Write $\b_1
=\sigma_1 \sigma_2$ and $\b_2 = \sigma_2^{-1}$.
It is immediate that $Q\sub P_{DD}$. Now we show the opposite. 
We have two cases:
\begin{enumerate}[label = Case \numbers.]
\item $\b$ or $\b^{-1}$ is $\sigma_2$-positive:
Then $\b = \sigma_2^p$ for some $p\in \ZZ\minus \left\{0\right\}$. For $p > 0$ we have
$\b^{-1}\in Q$, and for $p < 0$ we have $\b\in Q^{-1}$.

\item $\b$ is $\sigma_1$-positive: then 
there are
$m_i\in \ZZ$, $1\leq i\leq k$, 
such that
\begin{align}
\b &= \sigma_2^{m_1}\sigma_1 \sigma_2^{m_2} \sigma_1 \ldots \sigma_1 \sigma_2^{m_k}
\\
&= \b_2^{P_1} \b_1 \b_2^{P_2}\b_1 \ldots  \b_1 \b_2^{P_k}
\end{align}
for some 
$P_i\in \ZZ$.
Then we have
\begin{equation}
\b_2 \b_1^2 \b_2 = \b_1
\end{equation}
so we can cancel things and keep replacing $\b_1$ by this, until all exponents of $\b_2$
are positive, so $\b\in Q$.

\item $\b$ is $\sigma_1$-negative: so $\b^{-1}$ is $\sigma_1$-positive, so $\b^{-1}\in Q$
by case 2.
\end{enumerate}
Then this means $<_{DD}$ is an ordering on $B_n$, so it is isolated in
$\LO\left(G\right)$.
\end{proof}
