\marginnote{Lecture 3; January 28, 2020}

Let $<$ be an STO on a set $X$.
Let $\cB\left(X , <\right)$ be the group of $<$-preserving bijections $X\to X$.

\begin{thm}
$\cB\left(X, <\right)$ is always LO.
\label{thm:1.17}
\end{thm}

\begin{proof}
Let $\cless$ be a WO on $X$.
Let $f , g\in \cB\left(X , <\right)$ such that $f\neq g$. 
Write 
\begin{equation}
\left[f\neq g\right] = \left\{x\in X \st f\left(x\right)\neq g\left(x\right)\right\}\neq
\emp
\ .
\end{equation}
Let $x_0$ be the $\cless$-least element of $\left[f\neq g\right]$.
Define
\begin{equation}
f < g \iff f\left(x_0\right) < g\left(x_0\right) \ .
\end{equation}
Then we claim that this is an LO on $\cB\left(X , <\right)$.
Left-invariance is clear. To see this is a STO we need ``trichotomy''
and transitivity. Trichotomy is easy, and transitivity follows from the same argument as
the proof of \cref{thm:1.4}.
\end{proof}

\begin{exm}
Let $<$ be the standard order on $\RR$.
Then $\cB\left(\RR , <\right)$ consists of the orientation-preserving homeomorphisms $\RR
\to \RR$, written $\homeo_+\left(\RR\right)$.
\end{exm}

\begin{cor}
$\homeo_+\left(\RR\right)$ is LO.
\label{cor:1.18}
\end{cor}

\begin{rmk}
For $x\in \RR$, let $\cless_x$ be a WO on $\RR$ such that $x$ is the $\cless_x$-least
element of $\RR$.
Let $<_x$ be the LO on $\homeo_+\left(\RR\right)$ induced by $\cless_x$, as in the proof
of \cref{thm:1.17}. Given $x\neq y\in \RR$, there exists $g\in \homeo_+\left(\RR\right)$
such that
$g\left(x\right) > x$ and $g\left(y\right) < y$. But this means
\begin{align*}
g<_x 1 && g<_y 1
\ .
\end{align*}
which implies $<_x\neq <_y$.
Therefore $\homeo_+\left(\RR\right)$ has uncountably many LO's.
\end{rmk}

\begin{rmk}
It is a fact that the number of LO's on a group $G$ is either finite (and of the form
$2^n$) or uncountable.
\end{rmk}

\begin{cor}
A group $G$ is LO iff $G$ acts faithfully\footnote{Recall this means $g\left(x\right)= x$
for all $x\in X$ iff $g = 1$.} on a STO'd set $\left(X , <\right)$.
\end{cor}

\begin{proof}
$\left(\converse\right)$: This follows from \cref{thm:1.17}.

$\left(\implies\right)$: $G$ acts faithfully on $\left(G , <\right)$ by left
multiplication.
\end{proof}

\Cref{cor:1.18} implies that any subgroup of $\homeo_+\left(\RR\right)$ is LO. 
E.g. one can show that $F_2$ (the free group of rank $2$) is a subgroup of
$\homeo_+\left(\RR\right)$.
(This is another way to show that countable free groups are LO.)
In fact this characterizes countable LO groups.

\begin{thm}
Let $G$ be a countable group. 
Then $G$ is LO iff there exists an injective homomorphism $G\to \homeo_+\left(\RR\right)$.
\label{thm:1.20}
\end{thm}

\begin{proof}
$\left(\converse\right)$: This follows from \cref{cor:1.18}.

$\left(\implies\right)$: We actually prove something slightly stronger. This will follow
from \cref{thm:1.21}.
\end{proof}

\begin{thm}
Let $\left(G , <\right)$ be a countable group with an LO. Then there exists a LO on
$\homeo_+\left(\RR\right)$ and an order-preserving injective homomorphism
$\left(G , <\right)\to \left(\homeo_+\left(\RR\right) , <\right)$.
\label{thm:1.21}
\end{thm}

\begin{proof}[Sketch of proof]
Let $<$ be an LO on $G$. If $G = \left\{1\right\}$ this is immediate, so assume $G\neq
\left\{1\right\}$. Therefore it is infinite, since LO groups are torsion free.
Let $g_1 , g_2 , \ldots$ be some enumeration of the elements of $G$.

Define an embedding $e : G\to \RR$ by
$e\left(g_1\right) = 0$, and inductively by:
\begin{enumerate}[label = (\iii)]
\item If $g_{n+1} 
\begin{Bmatrix}
> \\ <
\end{Bmatrix}
g_i$ for all $1\leq i\leq n$, then set
\begin{equation}
e\left(g_{n+1}\right) = 
\begin{Bmatrix}
\max \left\{e\left(g_i\right) \st 1\leq i\leq n\right\} + 1 \\
\min \left\{e\left(g_i\right) \st 1\leq i\leq n\right\} - 1 
\end{Bmatrix}
\ .
\end{equation}
\item Otherwise let
\begin{align*}
g_l &= \max\left\{ g_i \st 1\leq i\leq n , g_i < g_{n+1}\right\} \\
g_r &= \min\left\{g_i \st 1\leq i\leq n , g_i > g_{n+1}\right\}
\end{align*}
and set
\begin{equation*}
e\left(g_{n+1}\right) = 
\frac{e\left(g_l\right) + e\left(g_r\right)}{2} \ .
\end{equation*}
\end{enumerate}
\begin{rmk}
\begin{enumerate}
\item $e$ is order-preserving, i.e. $a< b\implies e\left(a\right) < e\left(b\right)$.
\item $e\left(g_{n+1}\right)\in \ZZ$ iff (i) holds.
\item If $g > 1$ then $g^2 > g$ and $g^{-1} < g$. 
If $g < 1$ then $g^2 < g$ and $g^{-1} > g$, which implies $\ZZ\sub e\left(G\right) =
\Gamma$.
\item $G$ acts on $\Gamma$ by 
$g\left(e\left(a\right)\right) = e\left(ga\right)$. 
In fact, $G$ acts on $\left(\Gamma  , <\right)$ (where $<$ is the restriction of $<$ on
$\RR$) since
$e\left(a\right) < e\left(b\right)$ iff $a<b$ iff $ga < gb$ iff
$e\left(ga\right) < e\left(gb\right)$ iff $g\left(e\left(a\right)\right) <
g\left(e\left(b\right)\right)$.
\end{enumerate}
\end{rmk}

To see that
this action extends to an action of $G$ on $\RR$, we have a few steps.
\begin{enumerate}[label=Step \numbers:]
\item The action of $G$ on $\Gamma$ is continuous,
\label{item:step1}
\item The action of $G$ on $\Gamma$ extends to a continuous action of $G$ on $\bar
\Gamma$.
\item $\RR\minus \bar \Gamma$ is a countable $\dun$ of open intervals $\left(a_i ,
b_i\right)$;
the action of $G$ is defined on $\left\{a_i , b_i\right\}$; and extends to
$\left[a_i , b_i\right]$.
\end{enumerate}

Note, to ensure \ref{item:step1}, it is not enough to take $e$ to be an order-preserving
of $G$ in $\RR$. It must be continuous.

To define an LO on $\homeo_+\left(\RR\right)$ that restricts to the LO on $\Gamma$ from
$G$, first pick any $\gamma \in \Gamma$.
Then $g > 1$ (resp. $<1$) iff $g\left(\gamma\right) > \gamma$ (resp. $< \gamma$).
Let $\cless$ be a WO on $\RR$ such that $\gamma$ is the $\cless$-least element of $\RR$.
Then let $\dless$ be the LO on $\homeo_+\left(\RR\right)$ induced by $\cless$. Then
$g > 1$ (resp. $<1$) in $G$ iff $g\dgr 1$ (resp. $\dless$) in $\homeo_+\left(\RR\right)$.
\end{proof}

\section{Group rings}

Let $R$ be a ring (with $1$). 
\begin{itemize}
\item $a\in R$ is a \emph{unit} if there exists $b\in R$ such that $ab = ba = 1$.
\item $a\in R$ is a \emph{zero-divisor} if $a\neq 0$ and there exists $b\neq 0$ such that either
$ab = 0$ or $ba = 0$.
\item $a\in R$ is a \emph{non-trivial idempotent} if $a^2 = a$ but $a\neq 0$ and $a\neq 1$.
\end{itemize}
Let $G$ be a group and $R$ a ring. Then the \emph{$R$-group ring of $G$} consists of
formal sums:
\begin{equation}
RG \ceqq \left\{\sum r_g g \st g\in G , r_g\in R, r_G\neq 0\forall \text{ but f'tly many }
g\in G\right\}
\ .
\end{equation}
$RG$ is a ring with respect to the obvious operations. For $g\in G$ and $r\in R$ a unit,
then $rg$ is a unit in $RG$.
A unit in $RG$ is \emph{non-trivial} if it is not of this form.

\begin{rmk}
If $\tilde X\to X$ is a universal covering, then $\pi = \pi_1\left(X\right)$ acts on
$\tilde X$ so $H_*\left(\tilde X , \ZZ\right)$ is a $\ZZ \pi$-module.
\end{rmk}

\begin{thm}
Suppose $G$ has non-trivial torsion, and $K$ is a field of characteristic $0$.
\begin{enumerate}
\item $KG$ has zero divisors,
\item $KG$ has non-trivial units, 
\item $KG$ has non-trivial idempotents.
\end{enumerate}
\label{thm:1.22}
\end{thm}

\begin{proof}
Let $g\in G$ have order $n\geq 2$.
Define 
\begin{equation*}
\sigma = 1 + g + g^2 + \ldots + g^{n-1} \in KG
\ .
\end{equation*}
First notice that
\begin{equation}
g\sigma = \sigma
\label{eqn:*}
\end{equation}
which implies
$\left(1-g\right)\sigma = 0$
so we have zero divisors.

\eqref{eqn:*} also gives us that $\sigma^2 = n\sigma$.
Therefore
\begin{equation*}
\left(1-\sigma\right)\left(1 - \frac{1}{n-1} \sigma\right) = 1
\end{equation*}
so we have a nontrivial unit for $n > 2$. 
If $n = 2$, $1 - \sigma = -g$, but we still have:
\begin{equation}
\left(1 - 2\sigma\right)\left(1 - \frac{2}{3} \sigma\right) = 1 \ .
\end{equation}

Finally, we have that
\begin{equation}
\left(\frac{1}{n} \sigma\right)^2 = \left(\frac{1}{n^2}\right)\sigma^2 = \frac{1}{n}
\sigma
\end{equation}
so we have nontrivial idempotents.
\end{proof}

Note that the proof of (1) works even for $\ZZ G$.

\begin{rmk}
If $n\not\in \left\{2,3,4,6\right\}$ then 
$\ZZ G$ has nontrivial units.
This is a theorem of Higman.
\end{rmk}

\begin{exm}
For $n = 5$, 
\begin{equation}
\left(1 - g - g^4\right)\left(1 - g^2 - g^3\right) = 1 \ .
\end{equation}
\end{exm}

But what if $G$ is torsion free?
This brings us to the famous Kaplansky conjectures.

\begin{con}[Kaplansky]
If $G$ is torsion free and $K$ is a field, then:
\begin{enumerate}[label = \III]
\item (Units conjecture): $KG$ has no non-trivial units,
\label{KI}
\item (Zero-divisors conjecture): $KG$ has no zero divisors,
\label{KII}
\item (Idempotents conjecture): $KG$ has no non-trivial idempotents.
\label{KIII}
\end{enumerate}
\end{con}

\begin{rmk}
Clearly
\ref{KII} implies \ref{KIII} since $a^2 = a$ implies $a\left(a-1\right) = 0$, which by
\ref{KII} implies $a = 0$ or $a = 1$ which implies \ref{KIII}.
In fact they're all equivalent, but this is nontrivial to see.
\end{rmk}
