\marginnote{Lecture 2; January 23, 2019}
\begin{Proof}
Write $\phi : G\to Q$ and regard $H$ as $\ker \phi < G$.
Let $P_H$ (resp. $P_Q$) be positive cones for LO's on $H$ (resp. $Q$).
Define $P = \phi^{-1}\left(P_Q\right)\dun P_H$.
\begin{clm}
$P$ is a positive cone for an LO on $G$.
\end{clm}
\begin{proof}
We need to check (1) and (2) from \cref{lem:1.7}.
Let $g,h\in P$. Then we want to show $gh\in P$.
We have three cases.
\begin{enumerate}[label = (\abc)]
\item $g,h\in \phi^{-1}\left(P_Q\right)$: In this case
$\phi\left(g\right),\phi\left(h\right)\in P_Q$, so $\phi\left(gh\right) =
\phi\left(g\right)\phi\left(h\right)\in P_Q$. Therefore $gh\in \phi^{-1}\left(P_Q\right)$.

\item $g,h\in P_H$: In this case $gh\in P_H$.

\item $g\in \phi^{-1}\left(P_Q\right)$, $h\in P_H$:
Then $\phi\left(gh\right) = \phi\left(g\right)\in P_Q$, so $gh\in
\phi^{-1}\left(P_Q\right)$. Similarly $hg\in \phi^{-1}\left(P_Q\right)$.
\end{enumerate}

Now we need to check $P\dun P^{-1}\dun \left\{1\right\}$.
But this follows from the fact that:
\begin{equation}
G = \left(H\minus \left\{1\right\}\right) \dun \phi^{-1}\left(Q\minus
\left\{1\right\}\right)\dun \left\{1\right\}
= \phi^{-1}\left(P_Q\right)\dun \phi^{-1}\left(P_Q^{-1}\right)
\end{equation}
since $H\minus \left\{1\right\} = P_H\dun P_H^{-1}$.
\end{proof}

We leave (2) as an exercise.
[Hint: Recall $P$ is a positive cone for
BO on $G$ iff it is a conjugacy invariant cone for an LO.]
\end{Proof}

\section{Orderability of manifold groups}

\begin{exm}
Let $X^2$ be the Klein bottle.
This has fundamental group
\begin{equation}
K = \pi_1\left(X^2\right) = \lr{a,b \st b^{-1} a b = a^{-1}}
\ . 
\end{equation}
This fits in the SES:
\begin{equation}
\begin{cd}
1\ar{r}&
\ZZ\ar{r}&
K\ar{r}&
\ZZ\ar{r}&
1\\
& \lr{a}\arrow[equal]{u}&
b\arrow[mapsto]{r}&
gm&
\end{cd}
\end{equation}
which means $K$ is LO by \cref{thm:1.13}.

Note that $K$ is \emph{not} BO. We have that $a > 1$ iff $b^{-1} a b > 1$, but this is
$a^{-1}$, so $a^{-1} > 1$ which is a contradiction.

Notice that $\ZZ$ has exactly two LO's. The usual one, and the opposite.
Therefore, if we choose an LO on $\lr{a}$ and $K / \lr{a}$, this gives $4$ LO's on $K$
determined by:
\begin{enumerate}[label = (\iii)]
\item $a> 1$, $b > 1$;
\item $a> 1$, $b < 1$;
\item $a< 1$, $b > 1$;
\item $a< 1$, $b < 1$.
\end{enumerate}
\end{exm}

\begin{thm}
These are the only LO's on $K$.
\label{thm:1.14}
\end{thm}

\begin{Proof}
It suffices to show that each of these conditions determines a unique positive cone.
\begin{enumerate}[label = (\iii)]
\item $a> 1$, $b > 1$: 
\begin{clm}
$a^k < b$ for all $k\in \ZZ$.
\end{clm}
\begin{proof}
$b < a^k$ implies $a^{-k}b < 1$. But $a^{-k} b = ba^k$ and $b > 1$, so $b < a^k$ implies
$a^k>1$, which implies $ba^k > 1$ which is a contradiction.
\end{proof}
Note that every element in $K$ has a unique representative of the form $a^m b^n$
for $m,n\in \ZZ$.
\begin{clm}
$a^m b^n > 1$ iff either $n > 0$ or $n =0$ and $m > 0$.
\end{clm}
\begin{proof}
If $n = 0$, then this is clear. If $n > 0$, then $a^m b > 1$ for any $m$ by claim $1$ (for
$k = -m$). But we also know $b > 1$ which implies $b^n > 1$, so we get $a^m b^n > 1$ for
$n > 0$. On the other hand, if $m < 0$ then $a^m b^n = b^n a^{\pm m} = \left(a^{\mp m}
b^{-n}\right)^{-1}$. Then we know $a^{\mp m}b^{-n} > 1$ by the case above, so its inverse
is $<1$.
\end{proof}

If $<$ is an LO on $G$, and $\al : G\to G$ is an automorphism, then this induces an LO
$<_\al$ on $G$ given by:
$g <_\al h$ iff $\al\left(g\right) < \al\left(h\right)$.
Now notice that there are automorphisms $\al_1 , \al_2$ of $K$ such that
\begin{align}
\al_1\left(a\right) = a\ , && \al_1\left(b\right) = b^{-1} \\
\al_1\left(a\right) = a^{-1}\ , && \al_1\left(b\right) = b
\ .
\end{align}
In particular, $\al_1$ is given by
\begin{equation}
\lr{a,b \st b^{-1} a b = a^{-1}} \simeqq \lr{a , b \st bab^{-1} = a^{-1}}
\end{equation}
and similarly for $\al_2$.

Write $<_{\left(i\right)}$ for the unique LO on $K$ determined by (i). Then
$<_{\left(ii\right)}$ is induced by $<_{\left(i\right)}$ and $\al_1$,
$<_{\left(iii\right)}$ is induced by $<_{\left(i\right)}$ and $\al_2$,
and $<_{\left(iv\right)}$ is induced by $<_{\left(i\right)}$ and $\al_1 \al_2$.
\end{enumerate}
\end{Proof}

\begin{fact}
If $G$ has only finitely many LO's, then the number of LO's is of the form $2^n$.
\end{fact}

\begin{exr}
Show that for all $n\geq 0$ there exists a group $G$ with exactly $2^n$ LO's.
\end{exr}

\begin{cor}
For any LO on $K$, if $h\in \lr{a}$, $g\in K\minus \lr{a}$, and $g > 1$, then $g > h$.
\label{cor:1.15}
\end{cor}

\begin{proof}
It is sufficient to check this for the first LO, since the other three are determined by
the above automorphisms. 
Let $a > 1$, $b > 1$.
By claim 2 from above, we know $g = a^m b^n$ for $n > 0$. We now there is some $k$ such
that $h = a^k$, and therefore
\begin{equation}
h^{-1} g = a^{m-k} b^n > 1
\end{equation}
by claim $2$, so $g > h$.
\end{proof}

\section{Three-manifold groups}

Suppose $M$ is a closed, orientable, connected three-manifold.
Then we might ask if $\pi_1\left(M\right)$ is LO? BO?

Immediately we notice that not all such groups are.
If $M$ is a lens space, then $\pi_1\left(M\right) \simeqq \ZZ / n$ for $n > 1$, so this is
not LO.
More generally, for $\pi_1\left(M\right)$ nontrivial and finite is not LO.
Recall that if $M= M_1 \csum M_2$, then this implies $\pi_1\left(M\right)\simeqq
\pi_1\left(M_1\right) * \pi_1\left(M_2\right)$.
So, for example, if $M_1 \csum$ lens space, then $\pi_1\left(M\right)$ has torsion, so not
LO.

But at least some of them are.
Consider $M\simeqq T^3 = S^1\times S^1 \times S^1$. Then $\pi_1\left(M\right) = \ZZ^3$
is of course LO.
Similarly $M = \bcsum_n\left(S^1 \times S^2\right)\simeqq F_n$, so $\pi_1\left(M\right)$
is LO.

We will show that there exist (three-manifold) groups that are torsion-free, but not LO.

Let $p : T^2 \to X^2$ be a two-fold covering of the Klein bottle.
Recall that
\begin{equation}
K > p_*\left(\pi_1\left(T^2\right)\right) = \lr{a , b^2} \simeqq \ZZ\times \ZZ
\ .
\end{equation}
Let $N$ be the mapping cylinder of $p$, namely:
\begin{equation}
N = \left(T^2 \times I\right)\dun X^2 / \left( \left(x , 0\right)\sim p\left(x\right)
\forall x\in T^2\right) \ .
\end{equation}
The orientation reversing curve representing $b$ doesn't lift. So $N$ is orientable.
Note that $\p N \simeqq T^2$.
There is a strong deformation retraction $N\to X^2$, so $\pi_1\left(N\right)\simeqq K$.
Let $N_1$, $N_2$ be two copies of $N$.
Write
\begin{equation}
\pi_1\left(N_i\right) = \lr{a_i , b_i \st b_i^{-1} a_i b_i = a_i^{-1}}
\ .
\end{equation}
Notice that 
$\pi_1\left(\p N_i\right)\simeqq \ZZ\times \ZZ = \lr{a_i , b_i^2} < \pi_1\left(N_i\right)$.
Let $\phi : \p N_1 \to \p N_2$ be a homeomorphism.
Let $M_\phi = N_1 \un_\phi N_2$. This is a closed, orientable three-manifold. Therefore
\begin{equation}
\pi_1\left(M_\phi\right) = \pi_1\left(N_1\right)*_{\ZZ\times \ZZ}
\pi_1\left(N_2\right)\simeqq K_1 *_{\ZZ\times \ZZ} K_2 \ .
\end{equation}
Since $K$ is torsion-free, $\pi_1\left(M_\phi\right)$ is torsion-free.
But in fact we have the following theorem.

\begin{thm}
If $H_1\left(M_\phi\right)$ is finite, then $\pi_1\left(M_\phi\right)$ is not LO.
\label{thm:1.16}
\end{thm}

\begin{rmk}
We will see later that for $M$ a prime three-manifold with $H_1\left(M\right)$ infinite
has $\pi_1\left(M\right)$ LO.
\end{rmk}

\begin{proof}
$\phi$ is determined up to isotopy, so the resulting manifold $M_\phi$ depends only on
$\phi_* : H_1\left(\p N_1\right)\to H_1\left(\p N_2\right)$.
We know
\begin{align}
\ZZ\dsum \ZZ = \ZZ\lr{a_1 , 2b_1}
&&
\ZZ\dsum \ZZ = \ZZ\lr{a_2 , 2b_2}
\end{align}
so $\phi_*$ is given by some $2\times 2$ matrix
with $\ZZ$ coefficients
\begin{equation}
\begin{bmatrix}
p & r \\ q & s
\end{bmatrix}
\end{equation}
with determinant $ps - qr = \pm 1$. 
Specifically we have:
\begin{align}
\phi_*\left(a_1\right) &= pa_2 + 2qb_2 \\
\phi_*\left(2b_1\right) &= ra_2 + 2s b_2\ .
\end{align}
Now we have
$H_1\left(N_i\right) = \ZZ\dsum \ZZ_2$ with basis $b_i$ and $a_i$ respectively. 
Then $H_q\left(M_\phi\right)$ is presented by
\begin{equation}
A = \begin{bmatrix}
2 & 0 & 0 & 0 \\
0 & 0 & 2 & 0 \\
-1 & 0 & p & 2q \\
0 & -2 & r & 2s
\end{bmatrix}
\ .
\end{equation}
where we order the basis as $\left\{a_1 , b_1 , a_2 , b_2\right\}$.
Interchanging columns $2$ and $3$ we get
\begin{equation}
\det A = 4\abs{\det 
\begin{bmatrix}
0 & 2q \\ -2 & 2s
\end{bmatrix}} = 
16 \abs{q}
\ .
\end{equation}
Therefore $H_1\left(M_\phi\right)$ is finite iff $q\neq 0$ iff $\phi_*\left(a_1\right)\neq
\pm a_2$.

Suppose $\pi_1\left(M_\phi\right)$ is LO. Then we would get an induced LO on the common
boundary $\p N_1 = \p N_2$.
But there are only $4$ LO's on $\pi_1\left(N_i\right)$ (for $i \in \left\{1,2\right\}$).
By \cref{cor:1.15}, for any LO on $\pi_1\left(N\right)$, $\lr{a}$ is the unique
$\ZZ$-summand of 
$\pi_1\left(\p N\right) = \lr{a , b^2}$ such that if $h\in \lr{a}$ and $g\in \pi_1\left(\p
N\right)\minus \left\{1\right\}$, $g > 1$, then $g > h$.
Therefore $\phi_*\left(a_1\right) = \pm a_2$ which is a contradiction.
\end{proof}
