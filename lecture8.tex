\marginnote{Lecture 8; February 14, 2020}

\begin{rmk}
There is an analogue of this theorem for three-manifolds as well. This is related to the
the \emph{Whitehead manifold}, which is 
a contractible three-manifold not homeomorphic to $\RR^3$. Whitehead invented this as a
counterexample to his own theorem. Professor Cameron says this tells us it is okay to make
mistakes as long as you're the one to find the counterexample.
\end{rmk}

\begin{rmk}
Now \cref{thm:BH}
implies that if $G$ is locally indicable (and nontrivial) then $G$ is LO.
\end{rmk}

\begin{thm}
Let $S$ be a surface not homeomorphic to $\RP^2$. Then $\pi_1\left(S\right)$ is locally
indicable.
\end{thm}

\begin{proof}
Let $H < \pi_1\left(S\right)$, $H$ finitely generated and nontrivial.
Then we want to show it maps to $\ZZ$. The point is that there exists a connected covering
space $\tilde S \to S$ such that $\pi_1\left(\tilde S\right)\simeqq H$. By
\cref{thm:2.17}, $H\simeqq \pi_1\left(S_0\right)$ for $S_0$ a compact surface. Of course
$\pi_1\left(S_0\right)\neq \left\{1\right\}$ (since $H$ was).

Now we claim $S_0\not\simeqq \RP^2$. If it was, then $\tilde S\simeqq \RP^2$, so $S\simeqq
\RP^2$, which is a contradiction. Not by the classification of compact surfaces, there
exists an epimorphism $H_1\left(S_0\right) \surj \ZZ$, 
so we can just pre-compose with the map $\pi_1\left(S_0\right)
\surj H_1\left(S_0\right)$, so we get an epimorphism $H\surj \ZZ$.
\end{proof}

\begin{cor}
Let $S$ be a surface. Then $\pi_1\left(S\right)$ is LO 
if and only if $\pi_1\left(S\right)\neq \left\{1\right\}$ and $S\not\simeqq \RP^2$.
\label{cor:2.19}
\end{cor}

\begin{rmk}
\begin{enumerate}
\item If $S$ is the Klein bottle then $\pi_1\left(S\right)$ is locally indicable. But
$\pi_1\left(S\right)$ is not BO (there exists $a\in \pi_1\left(S\right)$ such that $a$ is
conjugate to $a^{-1}$).
This shows:
\begin{enumerate}
\item locally indicable and nontrivial does not imply BO, and
\item there is no analog of Burns Hale for BO.
\end{enumerate}
\item Locally indicable (and nontrivial) implies LO, but the converse is false. We will
see that there are three manifolds $M$ with $H_1\left(M\right)$ finite\footnote{So in
particular $\pi_1\left(M\right)$ is not locally indicable.} and
$\pi_1\left(M\right)$ LO. 

\item It can be shown that if $S$ is a non-compact surface, then $\pi_1\left(S\right)$ is
free. For example, $\RR\minus $ a cantor set has $\pi_1$ isomorphic to a free group on
a countably infinite number of generators. 
% But it has uncountably many ends.

\item It can be shown that $\pi_1\left(S\right) = 1$ if and only if $S\simeqq S^2$
or $D^2 \minus X$ for $X$ a closed subgroup of $S^1$.
\end{enumerate}
\end{rmk}

\begin{rmk}
Colin Adams is a 
knot theorist who gives lectures in different personas. E.g. a sleazy
real-estate agent selling property in hyperbolic space.
Once he attended a class posing as a student. He started heckling the lecturer, and
eventually the  lecturer said ``well if you know so much, you come teach the class!'' so
he did. Some of the students were responding to his heckling, saying ``shut up man, he's doing a
great job!'' so they were in for surprise when he revealed who he is.
\end{rmk}

\chapter{Three-manifolds}

Our three-manifolds will always be connected, orientable. They may have boundary and may
be non-compact. 
We will always be working in the PL or smooth category.

Let $M_1$ and $M_2$ be oriented $3$-manifolds with balls $B_i \sub \interior{M_i}$, $B_i\simeqq B^3$
for $i = 1 , 2$.
The \emph{connect sum} of $M_1$ and $M_2$ is the oriented manifold 
\begin{equation*}
M_1\csum M_2 = \left(M_1 \minus \interior{B_1}\right)\un_h \left(M_2 \minus \interior{B_2}\right)
\end{equation*}
for $h : \p B_1 \to \p B_2$ an orientation-reversing homeomorphism.
It turns out that $M_1 \csum M_2$ is well-defined (up to orientation-preserving
homeomorphism). The operation $\csum$ is associative, and commutative. Note that $M\csum S^3
\simeqq M$ for all $M$.
Also note that
\begin{equation*}
\pi_1\left(M_1 \csum M_2\right)\simeqq \pi_1\left(M_1\right) * \pi_1\left(M_2\right) \ .
\end{equation*}

We say
$M$ is \emph{prime} if $M\simeqq M_1 \csum M_2$ 
implies $M_1$ or $M_2\simeqq S^3$.

\begin{thm}[Kneser\cite{kneser},Milnor\cite{M_prime}]
Let $M$ be a compact, oriented $3$-manifold. Then 
\begin{equation*}
M\simeqq \bcsum_{i = 1}^n M_i
\end{equation*}
(orientation preserving (op))
where $M_i$ is prime (and not $\simeqq S^3$) for $1\leq i\leq n$. Moreover the $M_i$ are
unique up to order and orientation-preserving homeomorphism.
\end{thm}

Note $S^3$ corresponds to $n = 0$.

\begin{rmk}
In the same paper \cite{kneser} Kneser
proved some other things which relied on Dehn's lemma. So he was looking closer at Dehn's
proof, and found some holes. He wrote to Dehn who was on vacation to find out that he
agreed there was something fishy. Thus ensued a great correspondence between the two
trying to fix it. It was eventually fixed in \cite{P_dehn}.
\end{rmk}

For $M$ compact, and
\begin{equation*}
M\simeqq \bcsum_{i = 1}^n M_i
\end{equation*}
where the $M_i$ are prime, we have
\begin{equation*}
\pi_1\left(M\right)\simeqq \bfprod_{i = 1}^n \pi_1\left(M_i\right) \ .
\end{equation*}
So $\pi_1\left(M\right)$ is LO iff $\pi_1\left(M_i\right)$ is LO (for $1\leq i\leq n$).
This is also true for BO. 

\begin{exr}
Show $\pi_1\left(M\right)$ is locally indicable iff $\pi_1\left(M_i\right)$ is locally
indicable for all $1\leq i\leq n$.
[Hint: USe the Kurosh subgroup theorem.]
\end{exr}

The upshot is that for $M$ compact, to answer LO, BO, or locally indicable, we may assume $M$ is prime.

\begin{rmk}
There are noncompact three manifolds that cannot be expressed as $\#$ of prime manifolds.
\end{rmk}

$M$ is irreducible if every $S^2\sub M$ bounds a $B^3\sub M$.

\begin{fact}
$M$ is irreducible iff $M$ is prime
and not homeomorphic (op) to $S^1 \times S^2$.
\end{fact}

The point being that $S^1 \times S^2$ is prime.

\begin{thm}[Perelman\cites{perelman1,perelman2,perelman3}]
Let $M$ be a closed $3$-manifold with universal cover $\tilde M$. 
\begin{enumerate}
\item If $\pi_1\left(M\right)$ is finite, then $\tilde M\simeqq S^3$ and the action of
$\pi_1\left(M\right)$ on $S^3$ is as a subgroup of $\SO\left(4\right)$.

\item If $\pi_1\left(M\right)$ is infinite and $M$ is irreducible, then $\tilde M\simeqq
\RR^3$.
\end{enumerate}
\end{thm}

\begin{cor}[Poincar\'e conjecture]
If $M$ is closed and $\pi_1\left(M\right) = 1$, $M\simeqq S^3$.
\end{cor}

\begin{rmk}
We know $\pi_1\left(M\right)$ infinite implies $\tilde M$ is noncompact. 
Then $M$ irreducible
implies $\pi_2\left(M\right) = 0$ (as we will see soon) so by standard stuff, $\tilde M$
is contractible. 
But, there are contractible non-compact $3$-manifolds without boundary which are not
homeomorphic to $\RR^3$.
\end{rmk}

The $3$-manifolds with $\pi_1$ finite can be completely described. They're all Seifert
fiber spaces.

\begin{exm}
Let $p,q\in \ZZ$ such that $p\geq 2$ $\left(p,q\right) = 1$. 
Recall we have a $\ZZ / p$ action on $\CC^2$ by 
\begin{equation*}
\left(z,w\right)\mapsto \left(e^{2\pi i / p} z , e^{2\pi q i/ p} w\right)
\end{equation*}
Now the restriction of this action to $S^3$ is free, so we can quotient by it to get the
lens space $L\left(p,q\right)$. Then
\begin{equation*}
\pi_1\left(L\left(p,q\right)\right) = \ZZ / p \ .
\end{equation*}
Nonetheless, Alexander showed that $L\left(5,1\right) \not\simeqq L\left(5,2\right)$.
\end{exm}

\begin{thm}[Redemeister]
$L\left(p,q\right)$ is homeomorphic to $L\left(p , q'\right)$ 
iff either $q\cong q'\pmod{o}$ or $qq'\cong 1\pmod{p}$.
\end{thm}

The $\converse$ direction is easy.

\begin{thm}[Perelman\cites{perelman1,perelman2,perelman3}]
For $M$ and $M'$ closed three-manifolds, $M$ prime and not a lens space, then 
$\pi_1\left(M\right)\simeqq \pi_1\left(M'\right)$ implies $M'\simeqq M$.
\end{thm}

So ``prime three-manifolds are pretty much determined by their fundamental group''.

\begin{rmk}
The restriction to prime is necessary here.
Let $M$ be an oriented three-manifold such that $M$ is not homeomorphic (op) to $-M$. For
example, $M = L\left(3,1\right)$ or the Poincar\'e homology sphere.

Then $\pi_1\left(M\csum M\right)\simeqq \pi_1\left(M \csum \left(-M\right)\right)$, but by
prime decomposition, $M\csum M\not\simeqq M\csum \left(-M\right)$.
\end{rmk}
