\chapter{Orders on groups; basic definitions and properties}
\marginnote{Lecture 1; January 21, 2020}

The book for the course is \cite{book}.

Recall that a \emph{strict total order} (STO) on a set $X$ is a binary relation $<$
which satisfies:
\begin{enumerate}
\item $x < y$ and $y < z$ implies $x< z$;
\item $\forall x , y\in X$ exactly one of:
$x < y$, $y  <x$, $x = y$, holds.
\end{enumerate}

A \emph{left order} (LO) on a group $G$
is an STO such that $g < h$ implies $fg < fh$ for all $f\in G$.
$G$ is \emph{left-orderable} (LO) if there exists an LO on $G$. 
We similarly define a \emph{right order} (RO) 
and \emph{right orderability} (RO). 
A \emph{bi-order} (BO) on $G$ is an LO on $G$ that is also an RO.

\begin{rmk}
\begin{enumerate}
\item If $G$ is abelian, $<$ is a LO iff $<$ is an RO iff $<$ is a BO.
\item If $<$ is an LO on $G$, then $\cless$ defined by:
\begin{equation}
g\cless h \iff h^{-1} < g^{-1}
\end{equation}
is an RO on $G$.
Therefore $G$ is LO iff $G$ is RO.
We will stick to LO's.
\item For $H < G$, an LO (resp. BO) on $G$ induces an LO (resp. BO) on $H$.
\end{enumerate}
\end{rmk}

\begin{exm}
$\left(\RR , +\right)$ with the usual $<$ is BO. The subgroups $\ZZ < \QQ < \RR$ are also
BO.
\end{exm}

\begin{lem}
Let $<$ be an LO on $G$. Then
\begin{enumerate}
\item $g > 1$, $h > 1$ implies $gh > 1$;
\item $g > 1$ implies $g^{-1} < 1$;
\item $<$ is a BO iff 
$
\left(g < h \implies f^{-1} gf < f^{-1} h f \forall f\in G\right)
$ (i.e. $<$ is conjugation invariant).
\end{enumerate}
\label{lem:1.1}
\end{lem}

\begin{proof}
\begin{enumerate}
\item $h > 1$ implies $gh > g\cdot 1 g > 1$.
\item $g > 1$ implies $g^{-1} g > g^{-1}$ implies $1 > g^{-1}$.
\item $\left(\implies\right)$ is immediate. $\left(\converse\right)$: We need to show $<$
is a RO. $g < h$ implies $fg < fh$ implies $f^{-1} \left(fg\right)f <
f^{-1}\left(fh\right)f$ which implies $gf < hf$ as desired.
\end{enumerate}
\end{proof}

\begin{lem}
If $<$ is a BO on $G$, then
\begin{enumerate}
\item $g < h$ implies $g^{-1} > h^{-1}$;
\item $g_1 < h$, $g_2 < h_2$ implies $g_1 g_2 < h_1 h_2$.
\end{enumerate}
\label{lem:1.2}
\end{lem}

\begin{proof}
\begin{enumerate}
\item If $g < h$, then $g^{-1} g < g^{-1} h$, which implies $1 < g^{-1} h$, which implies
$1\cdot h^{-1} < g^{-1}$, which implies $h^{-1} < g^{-1}$.
\item $g_2 < h_2$ implies $g_1 g_2 < g_1 h_2 < h_1 h_2$.
\end{enumerate}
\end{proof}

\begin{wrn}
These don't necessarily true for LO's.
\end{wrn}

\begin{lem}
If $G$ is LO then it is torsion free.
\label{lem:1.3}
\end{lem}

\begin{proof}
Consider $g\in G\minus \left\{1\right\}$. If $g > 1$, then $g^2 > g > 1$, and similarly
for all $n\geq 1$, $g^n > 1$.
Similarly $g < 1$ implies $g^n < 1$ for all $n\geq 1$.
\end{proof}

So LO is not preserved under taking quotients (e.g. $\ZZ \to \ZZ / n$).

Consider an indexed family of groups $\left\{G_\lam \st \lam \in \Lam\right\}$.
Recall that the direct product
\begin{equation}
\prod_{\lam \in \Lam} G_\lam = \left\{
\left(g_\lam\right)_{\lam \in \Lam}
\right\}
\end{equation}
with multiplication defined co-ordinatewise.

Recall a \emph{well-order} (WO) on a set $X$ is a STO $\cless$ on $X$ such that if $A \sub
X$ and $A\neq \emp$ then there exists $a_0\in A$ such that $a_0 \cless a$ for all $a\in
A\minus \left\{a_0\right\}$.
Recall that the axiom of choice is equivalent to every set having a WO.

\begin{thm}
$G_\lam$ has a LO (resp. BO) for all $\lam \in \Lam$ iff $\prod_{\lam \in \Lam}G_\lam$
has a LO (resp. BO).
\label{thm:1.4}
\end{thm}

\begin{proof}
$\left(\converse\right)$: $G\lam < \prod_\lam G_\lam$ so we are finished.

$\left(\implies\right)$: Choose a WO $\cless$ on $\Lam$, and order $\prod_\lam G_\lam$
lexicographically. Let $g= \left(g_\lam\right)$, $h = \left(h_\lam\right)$, $g\neq h$.
Then $\lam_0$ be the $\cless$-least element of $\Lam$ such that $g_{\lam_0} \neq
h_{\lam_0}$. Then define $g < h$ iff $g_{\lam_0} < h_{\lam_0}$ (in $G_{\lam_0}$).
Then $<$ is an LO (resp. BO) on $\prod_\lam G_\lam$. Left (resp. left and right)
invariance is clear.
Now we show transitivity. Suppose $f < g$, $g < h$.
Let $\lam_0$ be the $\cless$-least element of $\Lam$ such that $f_{\lam_0} \neq
g_{\lam_0}$.
Let $\mu_0$ be the $\cless$-least element of $\Lam$ such that $g_{\mu_0}\neq h_{\mu_0}$.
\begin{enumerate}
\item $\left(\lam_0 \cleq \mu_0\right)$: Then $f_\lam = g_\lam = h_\lam$ for all $\lam
\cless \lam_0$.
Then $g_{\lam_0}$ is $<$ (resp. $=$) $h_{\lam_0}$ if $\lam_0 = \mu_0$ (resp. $\lam_0
\cless \mu_0$).
So $f_{\lam_0} < g_{\lam_0}\leq h_{\lam_0}$, and therefore $f_{\lam_0} < h_{\lam_0}$.

\item $\left(\mu_0 < \lam_0\right)$: This follows similarly.
\end{enumerate}
\end{proof}

Let $\sum_{\lam in \Lam} G_\lam$ be the \emph{direct sum of} $\left\{G_\lam\right\}$. Recall this
is the subgroup of $\prod_{\lam\in\Lam} G_\lam$ consisting of elements such that all but
finitely many co-ordinates are $1$.

\begin{cor}
$G_\lam$ is LO (resp. BO) for all $\lam \in \Lam$ iff $\sum_{\lam\in \Lam} G_{\lam}$ is LO
(resp. BO).
\label{cor:1.5}
\end{cor}

\begin{cor}
Free abelian groups are BO.
\label{cor:1.6}
\end{cor}

\begin{proof}
Free abelian groups on $\Lam$ are $\sum_{\lam \in \Lam} \ZZ$.
\end{proof}

Let $<$ be an LO on $G$.
The \emph{positive cone} $P = P_{<}$ of $<$ is $\left\{g\in G\st g > 1\right\}$.

\begin{lem}
Let $P$ be as above.
\begin{enumerate}
\item $g,h\in P$, implies $gh\in P$ (i.e. $PP\sub P$).
\item $G = P\dun P^{-1} \dun\left\{1\right\}$.
\item $<$ is a BO on $G$ iff $f^{-1} P f \sub P$ for all $f\in G$.
\end{enumerate}
\label{lem:1.7}
\end{lem}

\begin{proof}
\begin{enumerate}
\item This follows from \cref{lem:1.1} (1).
\item This follows from \cref{lem:1.1} (2).
\item This follows from \cref{lem:1.1} (3).
\end{enumerate}
\end{proof}

We say $P\sub G$ is a \emph{positive cone} if $P$ satsfies the conditions in
\cref{lem:1.7}.

\begin{lem}
Let $P\sub G$ be a positive cone. Then $g < h$ implies $g^{-1} h \in P$ defines a LO
$<$ on $G$ (With $P_< = P$).
\label{lem:1.8}
\end{lem}

\begin{proof}
$<$ is a STO, so:
\begin{enumerate}[label = (\iii)]
\item $f < g$, $g < h$ implies $f^{-1} g\in P$, $g^{-1} h\in P$, which implies (by the
first property) that $\left(f^{-1} g\right)\left(g^{-1} h\right)\in P$, which implies $f <
h$.

\item By the second property, for all $g,h\in G$ exactly one of the following holds:
$g^{-1} h\in P$, $g^{-1} h \in P^{-1}$, and $g^{-1}h = 1$.
Equivalently, $g<h$, $h < g$ (since $h^{-1} g\in P$), and $g = h$.
Now we show left invariance. $g < h$ implies $g^{-1} h\in P$, but $g^{-1} h = \left(g^{-1}
f^{-1}\right)\left(fh\right)$ which implies $fg < fh$.
\end{enumerate}
\end{proof}

\Cref{lem:1.7,lem:1.8} show that:
\begin{align}
\left\{\text{LO's on }G \right\}
&&\bij&&
\left\{\text{positive cones in } G\right\}
\\
\left\{\text{BO's on }G \right\}
&&\bij&&
\left\{\text{conjugacy-invariance positive cones in }G\right\}
\ .
\end{align}

Consider the free group of rank $n$, $F_n$.
\begin{thm}
$F_2$ is LO.
\end{thm}

\begin{Proof}[Proof by Sunic]
Write $F_2 = F\left(a,b\right)$.
$g\in F_2$ implies we can write it as a reduced word
\begin{equation}
\left(a^{m_1}\right)b^{n_1} \ldots a^{m_k}\left(b^{n_k}\right)
\end{equation}
for $k\geq 0$, $m_i , n_i\in \ZZ\minus \left\{0\right\}$.
Recall $1$ is the empty word, $k = 0$.
Let $e\left(g\right)$ be the number of syllables in $g$ with positive exponent, minus
the number of syllables in $g$ with negative exponent. Then define $j\left(g\right)$ so be
the number of $a^mb^n$'s in $f$, minus the number of $b^n a^m$s in $G$. So
$j\left(g\right) = 0$, or $\pm 1$.
For example:
\begin{align}
j\left(a^* \ldots a^*\right) &= 0 \\
j\left(b^* \ldots b^*\right) &= 0 \\
j\left(a^* \ldots b^*\right) &= 1 \\
j\left(b^* \ldots a^*\right) &= -1 \ .
\end{align}
Finally define
\begin{equation}
\tau\left(g\right) = e\left(g\right) + j\left(g\right) \ .
\end{equation}
Note that
\begin{align}
e\left(g^{-1}\right) = - e\left(g\right)
&&
j\left(g^{-1}\right)=  - j\left(g\right) \ .
\end{align}

\begin{lem}
If $g\neq 1$, then $\tau\left(g\right)\equiv 1\pmod{2}$.
\label{lem:1.10}
\end{lem}
\begin{proof}
$e\left(f\right)$ is congruent to the number of syllables mod $2$, and $j\left(g\right)$
is congruent to the number of syllables $+1$ mod $2$.
\end{proof}

\begin{lem}
$\abs{\tau\left(gh\right) - \tau\left(g\right) - \tau\left(h\right)} \leq 1$.
\label{lem:1.11}
\end{lem}
\begin{proof}
If $gh$ or $g$ or $h$ $=1$ we are done. So suppose $gh, g, h\neq 1$.
Clearly $e\left(gh\right)= e\left(g\right) + e\left(h\right)+ 
\begin{Bmatrix}
0\\ 1 \\ -1
\end{Bmatrix}
$.
Similarly:
\begin{equation}
j\left(gh\right) = j\left(g\right) + j\left(h\right) + 
\begin{Bmatrix}
0\\ 1 \\ -1
\end{Bmatrix}
\ .
\end{equation}
Therefore:
\begin{equation}
\abs{\tau\left(gh\right) - \tau\left(g\right) - \tau\left(H\right)}\leq 2
\end{equation}
so by \cref{lem:1.10}
we have
\begin{equation}
\abs{\tau\left(gh\right) - \tau\left(g\right)-  \tau\left(h\right)}\leq 1 \ .
\end{equation}
\end{proof}
\begin{rmk}
\Cref{lem:1.11} says that $\tau : F_2 \to \ZZ \left(< \RR\right)$ is what is called a
\emph{quasi-morphism}.
\end{rmk}

Define $P\sub F_2$ by
\begin{equation}
P = \left\{g\in F_2 \st \tau\left(g\right) > 0\right\} \ .
\end{equation}
Then $F_2 = P\dun P^{-1} \dun \left\{1\right\}$ by \cref{lem:1.10} and that
$\tau\left(g^{-1}\right) = -\tau\left(g\right)$.
Then $PP \sub P$ by \cref{lem:1.11} since
\begin{equation}
\tau\left(gh\right) \geq \tau\left(g\right) + \tau\left(h\right) - 1 \geq 1 \ .
\end{equation}
Therefore $P$ is a positive cone for a LO on $F_2$.
\end{Proof}

\begin{cor}
Any countable free group is LO.
\label{cor:1.12}
\end{cor}

\begin{proof}
A countable free group is a subgroup of $F_2$.
\end{proof}

\begin{rmk}
\begin{enumerate}
\item $\tau\left(a^{-1} b\right) = 1$, so $a^{-1}b > 1$, so $b > a$. On the other hand,
$\tau\left(ab^{-1}\right) = 1$, so $ab^{-1} > 1$, so $b^{-1} > a^{-1}$.
So $\tau$ does not define a BO on $F_2$.

\item We will see later that all free groups are LO.

\item Even later we will see that all free groups are BO.
\end{enumerate}
\end{rmk}

\begin{thm}
Let $1\to H\to G\to Q\to 1$ be a short-exact sequence of groups. Then
\begin{enumerate}
\item $H$, $Q$ LO implies $G$ is LO;
\item if $Q$ is BO and $H$ has a BO that is invariant under conjugation in $G$ then $G$
is BO.
\end{enumerate}
\label{thm:1.13}
\end{thm}
