\marginnote{Lecture 7; February 11, 2020}

\begin{proof}
$\left(\implies\right)$: This direction is immediate. 

$\left(\converse\right)$: Apply \cref{thm:BH}. Let $H < G$, $H\neq \left\{1\right\}$,
finitely generated. 
\begin{enumerate}[label = Case \numbers:]
\item $\left[G : H\right] = \infty$. By hypothesis, $H$
is indicable, so therefore (since $H$ is nontrivial) $G$ has quotient $\ZZ$.

\item $\left[G : H\right]$ finite. By hypothesis there exists an epimorphism $\phi :G\to
Q$ where $Q$ is LO. Therefore $Q$ is infinite, so $\phi\left(H\right)\neq
\left\{1\right\}$, 
(since $\left[Q : \phi\left(H\right)\right]$ is finite) and therefore $H$ has LO quotient
$\phi\left(H\right)$.
\end{enumerate}
\end{proof}

\begin{rmk}
It turns out that $G$ BO implies $G$ is locally indicable.
\end{rmk}

\begin{rmk}
We will eventually apply \cref{thm:2.13} to three-manifold groups.
But first we look at surfaces.
\end{rmk}

\section{Surface groups}

An $n$-manifold is a second-countable Hausdorff space $M$
such that for all $x\in M$ $x$ has a neighborhood $U$ such that 
either 
\begin{align*}
\left(U , x\right) \simeqq \left(\RR^n , 0\right)
&& \text{or} &&
\left(U , x\right) \simeqq \left(\RR^n_+ , 0\right) \ .
\end{align*}
Define the interior and boundary as:
\begin{align*}
\interior{M} &= \left\{x\in M \st x \text{ has a neighborhood of the first type}\right\}
\\
\p M &= \left\{x\in M \st x \text{ has a neighborhood of the second type}\right\}
\ .
\end{align*}
Note that $\left(\interior{M}\right) \cap \p M = \emp$.
Also  note that $\interior{M}$ is an $n$-manifold with empty boundary, and $\p M$ is an
$\left(n-1\right)$-manifold with empty boundary.
$M$ is \emph{closed} if $M$ is compact and $\p M = \emp$.

A \emph{triangulation} of $M$ is a homeomorphism $M\simeqq \abs{K}$, where $K$
is a locally finite simplicial complex.
Whether or not a manifold has a triangulation is a subtle 
question which wasn't settled until recently \cite{man}.

\begin{fact}
Every $n$-manifold has a triangulation for $n\leq 3$.
\end{fact}
This was shown for $n = 2$ in \cite{rado} and for $n=3$ in \cite{moise}.

For us, a \emph{surface} is a $2$-manifold.
There is the well-known classification of closed surfaces. 
In particular, they all either look like $S^2$, $T^2$, a connect sum of copies of $T^2$,
the projective plane $\PP^2$, or connect sums of copies of $\PP^2$.

There is also a classification of non-compact surfaces.

\begin{exm}
Consider the plane. Now attach handles as in \cref{fig:loch}.
This is an infinite genus non-compact surface.
Now consider the infinite genus surface in \cref{fig:jacob}. 
Are these homeomorphic?
See \cref{rmk:ends} for the answer.

\begin{figure}
\begin{overlay}
\pict{loch.pdf}{0.7}
\toptext{$\ldots$}{5,0}
\toptext{$\ldots$}{-5,0}
\end{overlay}
\caption{The Loch-Ness monster surface obtained by attached infinitely many handles to the
plane.}
\label{fig:loch}
\end{figure}

\begin{figure}
\begin{overlay}
\pict{jacob}{0.7}
\end{overlay}
\caption{The Jacob's ladder surface.}
\label{fig:jacob}
\end{figure}
\label{exm:loch_jacob}
\end{exm}

Now we consider the following question.
\begin{qn}
Which surface groups $\pi_1\left(S\right)$ are LO?
\end{qn}

We want to use \cref{thm:BH}, so 
we will consider finitely generated subgroups of surface groups.
First, recall the following.

\begin{lem}
If $M$ is a closed $n$-manifold, $N$ is a connected $n$-manifold, and $f : M\to N$ is an
injective map, then $f$ is a homeomorphism.
\label{lem:2.14}
\end{lem}

This uses the Jordan-Brouwer theorem for $S^{n-1}$s in $S^n$.
For $M$ compact, $N$ Hausdorff, it is enough to show $f$ is onto.

\begin{lem}
Let $S$ be a non-compact surface. Then $H_2\left(S\right) = 0$.
\label{lem:2.15}
\end{lem}

\begin{proof}
Triangulate $S$. Then we can get compact surfaces
$S_1 \sub S_2 \ldots \sub S$ such that
\begin{equation*}
S = \bun_{i = 1}^n S_i \ .
\end{equation*}
$\p S_i\neq \emp$ by \cref{lem:2.14}, so $S_i\simeq$ some $1$-complex. Therefore
$H_2\left(S_i\right) = 0$, for all $i$. And every $2$-cycle in $S$ is contained in some
$S_i$. Therefore $H_2\left(S\right) = 0$.
\end{proof}

\begin{lem}
Let $S$ be a surface, $\dd$ a circle component of $\p S$ such that
$\pi_1\left(\dd\right)\to \pi_1\left(S\right)$ is not injective. Then $S\simeqq D^2$.
\label{lem:2.16}
\end{lem}

\begin{proof}
For $S$ compact, this is true by the classification.
So let $S$ be non-compact.
Let $S^* = S\un D^2$ glued along $\dd$.
Then we have that the following commutes
\begin{equation*}
\begin{tikzcd}
\pi_1\left(\dd\right)\ar{r}\ar{d}{\simeqq}&
\pi_1\left(S\right)\ar{d}\\
H_1\left(\dd\right)\ar{r}&
H_1\left(S\right)
\end{tikzcd}
\ .
\end{equation*}
But now since
$\pi_1\left(\dd\right)\to \pi_1\left(S\right)$ is not injective, $H_1\left(\dd\right)\to
H_1\left(S\right)$ cannot be injective either. 
So now applying Mayer-Vietoris, we get
\begin{equation}
H_2\left(S^*\right) \simeqq \ker \left(H_1\left(\dd\right)\to H_1\left(S\right)\right) \ ,
\end{equation}
so by definition this is nonzero. But $S^*$ is noncompact, so this contradicts
\cref{lem:2.15}.
\end{proof}

\begin{rmk}
Have you answered the question from \cref{exm:loch_jacob} yet?
The answer has to do with the number of \emph{ends}, which is defined as follows.
Remove compact subsets and count the remaining components.
If we minimize the number of components, then this is the number of ends.
This is clearly a topological invariant.
The loch-ness monster has $1$, and Jacob's ladder has $2$.

We can also define the notion of the number of ends of a group.
As it turns out,
$e\left(G\right) = 0$ iff $G$ is finite.
Then, for example, we have
\begin{align*}
e\left(\ZZ\right) &= 2 \\
e\left(\ZZ^n\right) &= 1 \qquad \left(n\geq 2\right) \\
e\left(F_n\right) &= \infty
\ .
\end{align*}
Then it turns out that
for all $G$, $e\left(G\right) = 0,1,2$, or $\infty$.
\label{rmk:ends}
\end{rmk}

\begin{thm}[Compact core theorem for surfaces]
Let $S$ be a connected surface with $\pi_1\left(S\right)$ finitely generated. Then there
exists a compact connected $S_0 \linj{i} S$ such that 
$i_* : \pi_1\left(S_0\right) \to \pi_1\left(S\right)$ is an isomorphism. 
We call $S_0$ a \emph{compact core} of $S$.
\label{thm:2.17}
\end{thm}

\begin{proof}
Triangulate $S$. Let $\gamma_1 , \ldots , \gamma_n$ 
be simplicial loops in $S$ such that
$\left\{ \left[\gamma_1\right] , \ldots , \left[\gamma_n\right]\right\}$
are generators of $\pi_1\left(S\right)$. 
Let $N$ be a regular neighborhood of $\bun_{i = 1}^n \gamma_i$ in $S$.
$N$ is a compact surface with $\p N \neq \emp$ (and we can in fact assume it is connected)
and $\pi_1\left(N\right) \to \pi_1\left(S\right)$ is onto.

Let $S_0$ be $N$ union with
any disk components of $S$ cut along $\p N$.
$S_0$ is a compact surface, and $\pi_1\left(S_0\right)\to \pi_1\left(S\right)$ is onto.
If $\p S_0 = \emp$ then we are done since $S_0 = S$.

So suppose $\p S_0 \neq \emp$. Let $\dd$ be a component of $\p S_0$.
Since $\pi_1\left(S_0\right)\to \pi_1\left(S\right)$ is onto, $\dd$ separates $S$. 
(If not, there exists a loop $\gamma \sub S$ such that $\gamma \trans \dd$ is a single
point. Therefore $\gamma$ cannot be in $S_0$ but $\pi_1\left(S_0\right)\to
\pi_1\left(S\right)$ is onto.)

Let $S_1$ be the component of $S$ cut along $\dd$ such that $S_0\not\sub S_1$. By
definition of $S_0$ $S_1$ is not a disk.
Therefore by \cref{lem:2.16} $\pi_1\left(\dd\right)\to \pi_1\left(S_1\right)$ is
one-to-one.
If $S_0$ is a disk, then $\pi_1\left(S\right) = \left\{1\right\}$ and we are done. So
assume $S_0$ is not a disk. Then $\pi_1\left(\dd\right)\to \pi_1\left(S_0\right)$ is
injective.
So do this for all the boundary components $\dd$ of $S_0$. Then we see by Van-Kampen that
this is just a big free product:
\begin{equation*}
\pi_1\left(S\right)\simeqq 
\colim\left(
\begin{cd}
\pi_1\left(S_1\right)
\ar{drr}
&
\pi_1\left(S_2\right)
\ar{dr}
&
\pi_1\left(S_3\right)
\ar{d}
&
\ldots
&
\pi_1\left(S_k\right)
\arrow{dll}
\\
&& \pi_1\left(S_0\right) &&
\end{cd}
\right)
\end{equation*}
but by definition 
this means $\pi_1\left(S_0\right)\to \pi_1\left(S\right)$ is injective.
\end{proof}
