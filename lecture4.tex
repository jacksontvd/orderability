\marginnote{Lecture 4; January 30, 2020}

\begin{rmk}
Note that if $R$ is an integral domain (e.g. $\ZZ$) then $R$ is contained in its field of
fractions. In this case 
\cref{KI,KII} and \cref{KIII} 
for its field of fractions imply the corresponding versions of
\cref{KI,KII} and \cref{KIII} for $R$.
\end{rmk}
\begin{rmk}
We know this is true for LO groups. 
As we have seen,
we should think of LO as being a stronger version of torsion free.
\end{rmk}

\begin{thm}
If $G$ is LO then $KG$ satisfies
\cref{KI,KII} and \cref{KIII}.
\label{thm:1.23}
\end{thm}
\begin{proof}
Since \cref{KI} implies \cref{KIII} by the above remark
we show \cref{KI} and \cref{KII}. 

\cref{KI}: Suppose
\begin{equation}
\left(
\sum_{i = 1}^m \al_i g_i
\right)
\left(\sum_{j = 1}^n \b_j h_j\right) = 1
\end{equation}
with $m$, $n$ not both $1$, $\al_i , \b_j \neq 0\in K$, distinct $g_i\in G$, and distinct
$h_i\in G$.
Note this product can be rewritten as the following sum with $mn$ terms:
\begin{equation}
\label{eqn:star}
\sum_{i,j} \left(\al_i \b_j\right)\left(g_i h_j\right) \ .
\end{equation}
Assume WLOG that $h_1 < h_2 < \ldots < h_n$. Let $g_k h_l$ be a minimal element of
\begin{equation}
S = \left\{g_i h_j \st 1 \leq i\leq m , 1 \leq j\leq n\right\}
\sub G
\ .
\end{equation}
We know $h_1 < h_j$ for $j > 1$, so $g_k h_1 < g_k h_j$ for all $j > 1$.
Therefore $l = 1$.
Also $gh_1 = g'h_1$ which implies $g = g'$.
Therefore $g_k h_1$ is the unique
\begin{equation}
\left(k,1\right)\in \left\{\left(i,j\right) \st 1\leq i \leq m , 1\leq j \leq n\right\}
\end{equation}
such that $g_k h_1$ is a minimal element of $S$.

Similarly, there is a unique 
\begin{equation}
\left(r,n\right)\in \left\{\left(i,j\right) \st 1\leq i \leq m, 1\leq j\leq n\right\}
\end{equation}
such that $g_r h_n$ is a maximal element of $S$.
\begin{clm}
$g_k h_1 \neq g_r h_n$.
\end{clm}
If they were equal, then $r = k$, $n = 1$, so $m > 1$.
So $g_k h_1 = g_r h_1$, and therefore $g_r = g_k$. But this cannot be the case since they
are distinct by assumption.

This implies that \eqref{eqn:star}
has $\geq 2$ terms after cancellation, so it cannot be $1$.

\cref{KII}: Now suppose
\begin{equation}
\left(\sum_{i = 1}^m \al_i g_i\right) \left(\sum_{j = 1}^n \b_j h_j\right) = 0
\end{equation}
for $m , n \geq 1$. Then there is a unique minimal element and nonzero coefficient, which
means it is nonzero.
\end{proof}

\begin{con}[Isomorphism conjecture]
If $G$ is torsion free, then $\ZZ G \simeqq \ZZ H$ implies $G\simeqq H$.
\end{con}

\begin{rmk}
In \cite{hertweck} a finite counterexample to the conjecture for arbitrary groups was
provided, i.e. it is shown that
there exists finite $G$, $H$ such that $\ZZ G \simeqq \ZZ H$, $G\not\simeqq H$.
\end{rmk}

\begin{cor}[\cite{LR}]
If $G$ is LO, then $G$ satisfies the isomorphism conjecture.
\label{cor:1.24}
\end{cor}
\begin{proof}
\Cref{thm:1.23} implies that $\ZZ G$ has no nontrivial units. Call $\cU_{\ZZ G}$ the group
of units in $\ZZ G = \ZZ/ 2 \times G$. 
Suppose $\ZZ G \simeqq \ZZ H$. \Cref{thm:1.23} says that $\ZZ G$ has no $0$-divisors. This
implies $\ZZ H$ has no $0$-divisors, which means (by \cref{thm:1.22}) that $H$ is
torsion-free.
Now $H < \cU_{\ZZ H} \simeqq \cU_{\ZZ G} \simeqq \ZZ / 2 \times G$ which implies $H < G$
(since $H$ is torsion-free), which implies $H$ is LO (since $G$ is), which implies
$\cU_{\ZZ H} \simeqq \ZZ / 2 \times H$, which implies $\ZZ / 2 \times H \simeqq \ZZ / 2
\times G$ which implies $H\simeqq G$ (since $H$, $G$ are torsion free).
\end{proof}

\begin{rmk}
We might wonder if
it is ever the case that (for $G\neq 1$)
$\left(G * \ZZ\right) / \lr{\lr{w}} = 1$? This is known for 
$G$ torsion free \cite{klyachko}.
\end{rmk}
\begin{cexm}
If we consider the question of whether
we can ever have $\left(A* B\right) / \lr{\lr{w}} = 1$ for $A$, $B$ nontrivial, a
counterexample is given by:
\begin{equation*}
\ZZ / 2 * \ZZ / 3 / \left(a=b\right) \ .
\end{equation*}
\end{cexm}

\section{BO's on \texorpdfstring{$\ZZ \times \ZZ$}{Z x Z}}

Recall we have $2$ orders on $\ZZ$.
Consider a line of slope $\al$ in $\ZZ\times \ZZ$.
Then we have two cases.
\begin{enumerate}
\item $\al$ irrational: The associated positive cone is everything above the line.
Specifically, $P\sub \ZZ\times \ZZ$ is given by
\begin{equation}
P = \left\{\left(m,n\right) \st n > m\al\right\} \ .
\end{equation}
It is easy to check that this is a positive cone. 
This means there are uncountable many BO's on $\ZZ\times \ZZ$.

\item $\al$ rational: Notice that now
\begin{equation}
\left\{\left(m,n\right) \st n = m\al\right\} \simeqq \ZZ < \ZZ\times \ZZ
\ .
\end{equation}
Now let $P_0$ be one of the two positive cones on $\ZZ$. Then we can check that 
\begin{equation*}
P =P_0\dun \left\{
\left(m,n\right) \st n > m\al
\right\}
\end{equation*}
is a positive cone for $\ZZ \times \ZZ$.
\end{enumerate}

\begin{rmk}
\begin{enumerate}
\item (Up to reversal) these are all the BOs on $\ZZ \times \ZZ$. I.e. for $\al$ rational
we get two, and for $\al$ irrational we get $4$.
\item This generalizes in the obvious way to $\ZZ^n$.
\end{enumerate}
\end{rmk}

\section{BO's on \texorpdfstring{$\RR$}{the real line}}

Regard $\RR$ as a vector space on $\QQ$ with uncountable bases $\Lam$.
Recall $\Lam$ exists by the axiom of choice.
Therefore $\RR\sub \QQ^\Lam$. 
In particular it is the elements of $\QQ^\Lam$ with only finitely many non-zero
coordinates.
There are uncountable many WO's on $\Lam$, 
and each gives rise to a lexicographic BO on $\QQ^\Lam$.
This gives us uncountably many BOs on $\RR$.

\chapter{The space of left-orders on a group}

The basic idea is that since lefts orders are determined by positive cones, we can give
this space a topology. Consider a family of sets
$\left\{X_\lam \st \lam \in \Lam\right\}$.
Then write
\begin{equation*}
X = \prod_{\lam \in \Lam} X_\lam
\end{equation*}
and $\pi_\lam : X\to X_\lam$ for the projection. 
If $X_\lam$ is a topological space, then $X$ can be given the product topology. This is
the largest topology on $X$ such that $\pi_\lam$ is continuous for all $\lam \in \Lam$.
So $X$ has subbasis
\begin{equation}
\left\{\pi_\lam^{-1}\left(U_\lam\right) = U_\lam \times \prod_{\mu\neq \lam X_\mu}
\st U_\lam\sub X_\lam \text{ open}, \lam\in \Lam\right\}
\ .
\end{equation}

\begin{thm*}
If $X_\lam$ is compact for all $\lam \in \Lam$
then $\prod_{\lam \in \Lam} X_\lam$ is compact.
\end{thm*}

\begin{rmk}[Exercises]
\begin{enumerate}
\item $X_\lam$ Hausdorff (for all $\lam \in \Lam$) implies $\prod_{\lam \in \Lam} X_\lam$ is
Hausdorff.
\item A space $X$ is totally disconnected if the only nonempty connected subspaces are
singletons $\left\{x\right\}$ for $x\in X$.
This is equivalent to the connected components of $X$ all being $\left\{x\right\}$.
Show that $X_\lam$ totally disconnected (for all $\lam \in \Lam$)
implies $\prod_{\lam \in \Lam} X_\lam$ is totally disconnected.
\end{enumerate}
\end{rmk}

Let $X$ be a set, let $\cS\left(X\right)$ be the set of subsets of $X$ (i.e. the power
set). Then we have a correspondence:
\begin{align*}
\cS\left(X\right) && \bij && \left\{f : X\to \left\{0,1\right\}\right\}\\
\end{align*}
which sends:
\begin{align*}
A\sub X && \bij && f_A : X\to \left\{0,1\right\}
\end{align*}
where
\begin{equation*}
f_A\left(x\right) = 
\begin{cases}
1 & x\in A \\ 
0 & x\not \in A
\end{cases}
\ .
\end{equation*}

Give $\left\{0,1\right\}$ the discrete topology, and give 
\begin{equation*}
\cS\left(X\right) =
\left\{0,1\right\}^X = 2^X = \prod_{x\in X} \left\{0,1\right\}
\end{equation*}
the product topology. Note $\left\{0,1\right\}$ is a compact, Hausdorff,
totally-disconnected space, which means $\cS\left(X\right)$ is too.
For $x\in X$ let 
\begin{align*}
U_x &= \pi_x^{-1}\left(1\right) = \left\{A\sub X \st x\in A\right\} \\
V_x &= \pi_x^{-1}\left(0\right) = \left\{A\sub X \st x\not\in A\right\} \ .
\end{align*}
Note that $V_x = \cS\left(X\right)\minus U_x$ so $U_x$ and $V_x$ are open and closed.
Then
\begin{equation}
\left\{U_x \st x\in X\right\}\un \left\{V_x \st x\in X\right\}
\end{equation}
is a subbasis for $\cS\left(X\right)$.


