\marginnote{Lecture 6; February 6, 2020}

\begin{proof}
First notice $1\not\in P$ since
$1\not\in Q\left(F\right)$ for any finite $F\sub G\minus \left\{1\right\}$.

Now we show $g,h\in P$ implies $gh\in P$.
Let $F = \left\{g,h\right\}$. Then there are
$\e\left(g\right),\e\left(h\right)\in\left\{\pm 1\right\}$
such that 
\begin{align*}
S\left(g^{\e \left(g\right)} , h^{\e \left(h\right)}\right) \sub P
&&
S\left(g^{\e\left(g\right)} , h^{\e\left(h\right)}\right)^{-1} \cap P = \emp
\ .
\end{align*}
Therefore $\e\left(g\right) = \e\left(h\right) = +1$, which  implies
$gh\in S\left(g^{\e \left(g\right)} , h^{\e\left(h\right)}\right)
\sub P$.

Now we show $P\cap P^{-1} = \emp$. Let $g\in P$, and $F = \left\{g\right\}$. Therefore
$S\left(g\right)\sub P$, which means $S\left(g\right)^{-1} \cap P = \emp$, so $g^{-1}
\not\in P$.

Finally we show $P\dun P^{-1} G\minus \left\{1\right\}$. Take $g\in G$ such that $g\neq
1$. Let $F = \left\{g\right\}$. Then there exists $\e = \pm 1$ such that
$S\left(g^\e\right)\sub P$ (and $S\left(g^{-1}\right)\cap P = \emp$) which implies $g^\e
\in P$.
\end{proof}
\end{Proof}

\begin{rmk}
There  exists an analogue of this for BO.
\end{rmk}

\begin{thm}
$G$ is BO if and only if for all finite $F\sub G\minus \left\{1\right\}$ there is some $\e
: F\to \left\{\pm 1\right\}$ such that $1\not\in T\left(F , \e\right)$
where 
$T\left(F , \e\right)$ is the smallest semigroup which
\begin{enumerate}[label = (\iii)]
\item contains $S\left(F , \e\right)$, and
\item for all $g,h\in T\left(F , \e\right)$, $g,h,g^{-1}, g^{-1} h  g \in T\left(F ,
\e\right)$.
\end{enumerate}
\label{thm:2.4}
\end{thm}

\begin{exr}
Prove \cref{thm:2.4}.
\end{exr}

Let $P$  be a property of groups. A group $G$ is \emph{locally $P$} if and only if every
finitely generated subgroup of $G$ has property $P$.
(So $\loc\left(\loc\left(P\right)\right) \equiv \loc\left(P\right)$.)
$P$ is a \emph{local property} if $\loc\left(P\right)\implies P$.

\begin{thm}
$G$ is locally LO (resp. BO) if and only if $G$ is LO (resp. BO).
\label{thm:2.5}
\end{thm}

\begin{proof}
$\left(\converse\right)$: LO and BO are inherited by subgroups.

$\left(\implies\right)$: Let $G$ be a finite set contained in $G\minus \left\{1\right\}$.
Then $\lr{F} < G$ is finitely generated. 
$G$ $\loc\left(\LO\right)$ implies $\lr{F}$ is LO. Therefore there exists $\e$ such that 
\eqref{eqn:5*} holds (from \cref{thm:2.3}). This is true for all $F$, so $G$ is LO by
\cref{thm:2.3}.
The argument for BO is similar, using \cref{thm:2.4} instead.
\end{proof}

\begin{cor}
An abelian group is BO iff it is torsion free.
\label{cor:2.6}
\end{cor}

\begin{proof}
$\left(\implies\right)$: This follows from \cref{lem:1.3}.

$\left(\converse\right)$ $G$ is LO iff $G$ is $\loc\left(\LO\right)$.
For $H$ finitely generated inside of torsion free $G$, then $H\simeqq \ZZ^n$, so it is LO.
\end{proof}

\begin{cor}
An arbitrary free group is LO.
\label{cor:2.7}
\end{cor}

\begin{proof}
Let $F$ be a free group. For $H$ a finitely generated subgroup of $F$, $H\simeqq F_n$ for
some $n$.
Then $H$ is LO by \cref{cor:2.7}, so $F$ is LO by \cref{thm:2.5}.
\end{proof}

\begin{thm}
Let $\left\{G_\lam\right\}_{\lam\in \Lam}$ be a collection of groups. 
Then $G_\lam$ is LO for all $\lam \in \Lam$ if and only if $\bfprod_{\lam \in \Lam}
G_\lam$ is LO.
\label{thm:2.8}
\end{thm}

\begin{proof}
$\left(\converse\right)$: $G_\lam < \bfprod_{\lam \in \Lam} G_\lam$.

$\left(\implies\right)$: There exists a homomorphism 
\begin{equation*}
\begin{cd}
G = \bfprod_{\lam \in \Lam}G_\lam \ar{r}{\phi}&
\prod_{\lam \in \Lam} G_\lam \\
g_\lam \ar[mapsto]{r}&
\left(1 , \ldots , 1 , g_\lam , 1 , \ldots\right)
\end{cd}
\ .
\end{equation*}
So we get a SES
\begin{equation}
\begin{cd}
1\ar{r}&
H\ar{r}&
\bfprod_\Lam G_\lam \ar{r}{\phi}&
\prod_\Lam G_\lam \ar{r}&
1
\end{cd}
\end{equation}
where $H = \ker \phi$.
By the Kurosh subgroup theorem
\begin{equation*}
H = \left(\bfprod_\mu H_\mu\right) * F
\end{equation*}
where $H_\mu$ is a subgroup of a conjugate of $G_{\lam_\mu}$ in $G$, and $F$ is a free
group. But $H = \ker \phi$, and $\restr{\phi}{G_\lam}$ is injective for all $\lam \in
\Lam$.
Therefore for all $\lam \in \Lam$ and $g\in G$ we have
$H\cap g^{-1} G_\lam g = \left\{1\right\}$. Therefore $H = F$.

But now $G_\lam$ LO for all $\lam \in \Lam$ implies
$\prod_{\lam \in \Lam} G_\lam$
is LO by \cref{thm:1.4}, and $F = H$ is LL by \cref{cor:2.7}, so $G$ is LO by 
\cref{thm:1.13}.
\end{proof}

Let $P$ be a property of groups. A group $G$ is residually $P$, $\res\left(P\right)$, if
and only if for all $g\in G\minus \left\{1\right\}$ there exists an epimorphism $\phi :
G\to H$ such that $H$ has property $P$, and $\phi\left(g\right)\neq 1$.
\begin{rmk}
Note that
$P$ implies $\res\left(P\right)$, and  $\res\left(\res \left(P\right)\right)$ implies
$\res\left(P\right)$.
\end{rmk}

We say $P$ is a \emph{residual property} if and only if $\res\left(P\right)$ implies $P$.

\begin{exm}
Finiteness is not a residual property. 
E.g. $\ZZ$ is $\res\left(\text{finite}\right)$.
\end{exm}

\begin{lem}
If $P$ is closed under taking subgroups and direct products, then $P$ is a residual
property.
\label{lem:2.9}
\end{lem}

\begin{cor}
LO and BO are residual properties.
\label{cor:2.10}
\end{cor}

\begin{proof}[Proof of \cref{lem:2.9}]
Suppose $G$ is $\res\left(P\right)$. Then for all $g\in G\minus \left\{1\right\}$
there is an epimorphism $\phi_g : G\to H_g$ such that 
$H_g$ has $P$, and $\phi_g\left(g\right)\neq 1$.
The collection of these $\left\{\phi_g \st g\in G\minus \left\{1\right\}\right\}$ induces
a homomorphism
\begin{equation*}
\phi : G \to \prod_{g\in G\minus \left\{1\right\}} H_g
\ .
\end{equation*}
Then this is injective, and $\phi_g\left(g\right)\neq 1$.
$H_g$ has $P$ for all $g\in G\minus \left\{1\right\}$. Therefore $\prod_{g\in G\minus
\left\{1\right\}} H_g$ has $P$.
But 
\begin{equation*}
G\simeqq \phi\left(G\right)< \prod_{g\in G\minus \left\{1\right\}} H_g
\end{equation*}
so $G$ has $P$.
\end{proof}

\begin{rmk}
Residual properties are related to areas of  active research. For example the
geometrization conjecture is related to residual finiteness of $3$-manifolds.
\end{rmk}

\begin{rmk}
Let $G$ be a group. Let $\FQ\left(G\right)$ consist of the finite quotients of $G$.
Then the following is  an open question. Let $F_2$ be a free group of rank $2$. If $G$ is
a residually finite group  such that $\FQ\left(G\right) = \FQ\left(F_2\right)$ is
$G\simeqq F_2$?
Note that $\FQ\left(F_2\right)$ consists of the finite groups generated by two elements.
So this is really quite concrete.

Another open question is if $G_1$ and $G_2$ are residually finitely presented, then does
$\FQ\left(G_1\right) = \FQ\left(G_2\right)$ imply $G_1 \simeqq G_2$?
\end{rmk}

\begin{exm}
$\LO\left(\ZZ^n\right)$ and $\LO\left(F_n\right)$ are both the cantor set.
\end{exm}

\begin{exm}
Let $B_n$ denote the braid group.
As it turns out
$\LO\left(B_n\right)$ has isolated points \cite{dubrov}.
\end{exm}

The following is a strengthening of the fact that LO is a local property. 

\begin{wrn}
At this point it is convenient to make the convention that $\left\{1\right\}$ is
\emph{not} LO.
\end{wrn}

\begin{thm}[Burns-Hale]
$G$ is LO iff every non-trivial finitely generated subgroup $H< G$ has an LO quotient.
% \label{thm:2.11}
\label{thm:BH}
\end{thm}

\begin{proof}
$\left(\implies\right)$: $G$ is LO implies $H$ is LO.

$\left(\converse\right)$: $F = \left\{g_1 , \ldots , g_n\right\}\sub G\minus
\left\{1\right\}$ for $n\geq 1$.
We show by induction on $n$ that the condition on $F$ in \cref{thm:2.3} holds.
Let $n = 1$. Then $\lr{g_1}$ has an LO quotient by assumption. 
Therefore $g_1$ has infinite  order, so $1\not\in S\left(g_1\right)$. 
Now suppose $n > 1$. By assumption, there exists a nontrivial homomorphism 
$\phi : \lr{g_1 , \ldots , g_n} \to L$ where $L$ is LO.
For some $m$ there exists
\begin{equation*}
\phi\left(g_i\right) = 
\begin{cases}
+1 & 1 \leq i\leq m \\
-1 & m< i \leq n
\end{cases}
\end{equation*}

By the induction hypothesis there exists $\e_1 , \ldots , \e_m\in \left\{\pm 1\right\}$
such that $1\not \in S\left(\left\{g_i^{\e_i}  \st 1\leq i\leq m\right\}\right)$.
Let $<$ be an LO on $L$. Define  $\e_i \in \left\{\pm 1\right\}$ ($m < i \leq n$) 
so that
\begin{equation}
\phi\left(g_i^{\e_i}\right) > 1
\end{equation}
Then $1\not \in S\left(\left\{g_i^{\e_i} \st 1\leq i\leq n\right\}\right)$.
\end{proof}

A group $G$ is \emph{indicable} if either $G = \left\{1\right\}$ or there is an
epimorphism $G\to \ZZ$.

\begin{cor}
$G$ is locally indicable implies $G$ is LO.
\label{cor:2.12}
\end{cor}

\begin{rmk}
Free groups are $\loc\left(\text{indicable}\right)$ so this gives  another proof that free
groups are LO.
\end{rmk}

\begin{rmk}
Note that $G$ having an LO quotient does not imply 
$G$ is LO. 
\begin{cexm}
$\ZZ * \ZZ / 2$ has LO quotient, but is not LO.
\end{cexm}
We do however have:
\end{rmk}

\begin{thm}
Let $G$ be a group such that every finitely generated subgroup of infinite index is
indicable. Then $G$ is LO if and only if $G$ has an LO quotient.
\label{thm:2.13}
\end{thm}
