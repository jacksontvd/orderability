\marginnote{Lecture 5; February 4, 2020}

\begin{lem}
Suppose $B\sub X$. Then
\begin{align*}
\left\{A\sub X \st B\not \sub A\right\}
&&
\left\{A\sub X \st A\cap B\neq \emp\right\}
\end{align*}
are open subsets of $\cS\left(X\right)$.
\label{lem:2.1}
\end{lem}

\begin{proof}
\begin{align*}
\left\{A\sub X \st B\not \sub A\right\} &=
\bun_{b\in B} \left\{A\sub X \st b\not\in A\right\} = \bun_{b\in B} V_b
\end{align*}
so it is open. The argument for the other set is similar.
\end{proof}

If $G$ is a group, let 
\begin{equation}
\LO\left(G\right) = \left\{\text{positive cones}\sub G\right\}\sub \cS\left(G\right)
\end{equation}
and equip it 
with the subspace topology. 
We call this the \emph{space of left-orders on $G$}.

\begin{exm}
$\LO\left(\ZZ\right) = \pt \dun \pt$.
$\LO\left(\ZZ\times \ZZ\right)$ is the cantor set.
\end{exm}

\begin{thm}
$\LO\left(G\right)$ is closed in $\cS\left(G\right)$ and hence compact.
\label{thm:2.2}
\end{thm}

\begin{proof}
We show $\cS\left(G\right)\minus \LO\left(G\right)$ is open. Suppose $A\in
\cS\left(G\right)\minus \LO\left(G\right)$, i.e. $A\sub G$ is not a positive cone. 
So either:
\begin{enumerate}[label = (\iii)]
\item $\exists g,h\in A$ such that $gh\not\in A$ or
\item $\exists g\in G$ such that $g,g^{-1} \in A$ or
\item $1\in A$ or
\item $\exists g$, $g\neq 1$ such that $g\not\in A$ and $g^{-1} \not\in A$.
\end{enumerate}
Now the point is that these are open conditions since we can write them in terms of the
$U_x$'s and $V_x$'s.
In particular:
\begin{align*}
\left(i\right) \iff
A\in U_g\cap U_h \cap V_{gh}
&&
\left(ii\right)\iff A\in U_g\cap U_{g^{-1}} \\
\left(iii\right)\iff A\in U_1 
&&
\left(iv\right) \iff A\in \bun_{g\neq 1} \left(V_g \cap V_{g^{-1}}\right) \ .
\end{align*}
Therefore $\LO\left(G\right)$ is compact, Hausdorff, and totally disconnected.
\end{proof}

Similarly one can define the space of biorders on $G$, $\BO\left(G\right)$, to be 
the set of conjugation invariant positive cones in $G$.

\begin{exr}
Show that $\BO\left(G\right)$ is closed inside of $\LO\left(G\right)$.
\end{exr}

Therefore $\BO\left(G\right)$ is compact, Hausdorff, and totally disconnected.

\section{The cantor set}

The cantor set $C\sub I\sub \RR$ is defined as follows.
First write
\begin{align*}
C_1 &= \left[0,1/3\right] \un \left[2/3 , 1\right] \\
C_2 &= \left(\left[0,1/9\right]\un \left[2/9, 1/3\right]\right)\un
\left(\left[2/3, 7/9\right]\un \left[8/9, 1\right]\right) \\
\ldots &
\end{align*}
then define
\begin{equation}
C = \binter_{n = 1}^\infty C_n \ .
\end{equation}
The idea is that we keep removing the middle thirds. 

$C$ is uncountable, totally-disconnected, closed in $I$. 
Therefore it is also compact and Hausdorff.
This is a very surprising example.
We can easily write down something uncountable and totally-disconnected, such as the
irrationals, but they do not form a compact set.

Any $x\in I$ has a ternary expansion:
\begin{equation*}
x = 0.\, x_1 x_2 \ldots = \sum_{n = 1}^\infty \frac{x_n}{3^n}
\end{equation*}
which is unique up to:
\begin{equation*}
\ldots x_k 22\ldots = \ldots \left(x_{k+1}\right)0 0 \ldots
\ .
\end{equation*}
Now notice
\begin{align*}
x_1 = 1 && \iff && x\in \left(1/3 , 2/3\right)
\end{align*}
with the convention that 
\begin{equation*}
\frac{1}{3} = 0.022 \ldots
\ .
\end{equation*}
Similarly (with the same convention) we have
\begin{align*}
x_1 \neq 1, x_2 = 1 && \iff &&
x\in \left(1/9 , 2/9\right)\un\left(7/9 , 8/9\right)
\end{align*}
and so on. Then 
\begin{equation}
C = \left\{x\in I \st x= 0.\, x_1 x_2 \ldots \st \forall n, x_n = 0 \text{ or } 2\right\}
\ .
\end{equation}
Now give $\left\{0,2\right\}^\NN$ the product topology. 

\begin{exr}
Show that the map sending
\begin{equation}
0. x_1 x_2 \ldots \mapsto \left(x_1 , x_2 , \ldots\right)
\end{equation}
defines a homeomorphism
\begin{equation}
C\lto{\simeqq} \left\{0,2\right\}^\NN
\ .
\end{equation}
\end{exr}

Now recall that $\LO\left(G\right)$ is compact in $\left\{0,1\right\}^G$, so if $G$
is countable, then $\LO\left(G\right)$ is homeomorphic to a subspace of $C$.

We say $x\in X$ is \emph{isolated} if $\left\{x\right\}$ is open. 
We say $X$ is perfect if it has no isolated points.
As it turns out, the Cantor set is perfect.

\begin{thm*}
If $X$ is a compact, totally-disconnected, and perfect metric space, then $X\simeqq C$.
\end{thm*}

Therefore, if $G$ is countable, $\LO\left(G\right)\neq \emp$, and has no isolated points,
then $\LO\left(G\right)\simeqq C$.

\begin{exm}
In 2004
\cite{sikora} it was shown
that if $n > 1$ then $\LO\left(\ZZ^n\right) = \BO\left(\ZZ^n\right)\simeqq
C$.
\end{exm}

\begin{exm}
In 1985 \cite{mc} it was shown that
$\LO\left(F_n\right)\simeqq C$.
It is unknown if $\LO\left(F_n\right)$ has isolated points.
\end{exm}

\begin{rmk}
As it turns out, the braid group is LO. The first proof of this fact was 
not topological, so topologists started to think of a topological proof. When
someone asked
Thurston, he said 
``of course the braid group is left-orderable!''
\end{rmk}

If $X\sub G$, let $S\left(X\right)$ be the semigroup generated by $X$ in $G$. This is the same 
as the non-empty product of elements in $X$.
There is a characterization of left orderability in terms of finite subsets of $G$.

\begin{thm}
$G$ is LO iff for all finite $F\sub G\minus \left\{1\right\}$, there exists $\e : F\to
\left\{\pm 1\right\}$ such that
\begin{equation}
1\not \in S\left(\left\{f^{\e\left(f\right)} \st f\in F\right\}\right) \left(
= S\left(F , \e\right)
\right)
\ .
\label{eqn:5*}
\end{equation}
\label{thm:2.3}
\end{thm}

\begin{rmk}
It follows from this that, given a solution to the word problem in $G$, there exists a
machine such that if $G$ is not LO, the machine will eventually 
tell you that.
Nathan Dunfield has an explicit algorithm for three-manifold groups.
\end{rmk}

\begin{rmk}
If we take the $n$-fold cyclic branch cover of the knot $5_2$, then 
we can consider
$\pi_1\left(\Sigma_n \left(5_2\right)\right)$. For $n = 2$, this is a lens space
so $\pi_1$ is finite. 
It is also not LO for $n = 3,4,$ and $5$. But it is unknown for $n = 6,7$, and $8$.
(If the $L$-space conjecture is true,\footnote{Which is looking quite likely. It has been
checked for something like three-hundred thousand manifolds.}
then it should be LO for these values of $n$.)
For $n\geq 9$ it is known to be LO.
\end{rmk}

\begin{Proof}
$\left(\implies\right)$:
Define
\begin{align*}
\e\left(f\right) = 
\begin{cases}
+1 & f > 1 \\
-1 & f < 1
\end{cases}
\ .
\end{align*}

$\left(\converse\right)$: Let $F\sub G\minus \left\{1\right\}$ be finite, $\e : F\to
\left\{\pm 1\right\}$. Define
\begin{equation*}
Q\left(F , \e\right) \ceqq \left\{Q\sub G\minus \left\{1\right\} \st S\left(F , \e\right)
\sub Q, S\left(F , \e\right)^{-1} \cap Q = \emp\right\} \ .
\end{equation*}
Note that $Q\left(F , \e\right)\neq \emp$ iff \eqref{eqn:5*} holds.
Let
\begin{equation*}
Q\left(F\right) = \un_\e Q\left(F , \e\right) \ .
\end{equation*}
Note this is a finite union.
\begin{clm}
$Q\left(F\right)$ is closed in $\cS\left(G\right)$.
\end{clm}
\begin{proof}
It is sufficient to show that $Q\left(F , \e\right)$ is closed, i.e. 
$\cS\left(G\right)\minus Q\left(F , \e\right)$ is open. Suppose $A\sub G$, $A\not \in
Q\left(F , \e\right)$ i.e. either $1\in A$, or $S\left(F , \e\right)\not \sub A$, or
$S\left(F , \e\right)^{-1} \cap A \neq \emp$. These conditions are all open by
\cref{lem:2.1}. 
\end{proof}
Note that if $F\sub F'$, then
\begin{equation}
S\left(F , \restr{\e'}{F'}\right)\sub S\left(F' , \e'\right)
\end{equation}
and therefore
\begin{equation}
Q\left(F'\right) \sub Q\left(F\right) \ .
\end{equation}
Let $F_1, F_2 , \ldots , F_n$ be finite subsets of $G\minus \left\{1\right\}$.
Then
\begin{equation*}
\binter_{i = 1}^n Q\left(F_i\right) \sups Q\left(F_1 \un F_2 \un \ldots \un F_n\right)\neq
\emp
\end{equation*}
since \eqref{eqn:5*} holds.
This means $\left\{Q\left(F\right)\right\}$ has the \emph{finite intersection property}
(FIP) and each one is closed. Therefore, since $\cS\left(G\right)$ is compact, 
\begin{equation*}
\binter_{F\sub G\minus \left\{1\right\}\text{ finite}} Q\left(F\right) \neq \emp
\ .
\end{equation*}
So let $P\in \binter Q\left(F\right)$. 
\begin{clm}
$P$ is a positive cone for $G$.
\end{clm}
